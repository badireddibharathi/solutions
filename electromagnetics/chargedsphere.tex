% !TEX root = main.tex

\section*{chargedsphere}
Radius of sphere is $R$.\\
Charge density is uniform  and $\rho(r)=kr$\\
Electric field is to be calculated at location $\vec{r}=z\hat{z}$.
\subsection*{Case $(z>R)$} 
The location of point is outside the sphere.
Consider the sphere to be made of concentric shells of radius $r$ and thickness $\dd r$. Consider a radius vector making an angle $\theta$ with $z$ axis. All the points that satisfy this condition are on a circle of radius $r\sin{\theta}$. Consider another radius vector making an angle $\theta+\dd\theta$ with $z$ axis. All the points that satisfy this condition are on a circle of radius $r\sin{(\theta+\dd\theta)}$. The element is the area of sphere between these circles.The width of the element is $Rd\theta$. Area of the element is $2\pi(r\sin{\theta})(r\dd\theta)=2\pi\,r^2\sin{\theta}\,d\theta$\\
Charge on the element is $\dd Q=2\pi k\,r^3\sin{\theta}\dd r\dd\theta$.\\
Electric field due to a charge $Q$ spread uniformly on a ring of radius $R$ at a distance $z$ along the axis of the ring is $$\vec{E}(z)=\frac{1}{4\pi\epsilon_0}\frac{Qz}{(R^2+z^2)^{3/2}}\hat{z}$$
Using the above result,
Electric field due to the above element is $$d\vec{E}=\frac{1}{4\pi\epsilon_0}\frac{(2\pi k\,r^3\sin{\theta}\dd r\dd\theta)(z-R\cos{\theta})}{((R\sin{\theta})^2+(z-R\cos{\theta})^2)^{3/2}}\hat{z}$$
Integrating over the above result from $\theta=0$ to $\theta=\pi$
$$\vec{E}(z)=\int\,d\vec{E}(z)=\int_{0}^{\pi}\frac{1}{4\pi\epsilon_0}\frac{(2\pi\sigma\,R^2\sin{\theta}\,d\theta)(z-R\cos{\theta})}{((R\sin{\theta})^2+(z-R\cos{\theta})^2)^{3/2}}\hat{z}$$
$$\vec{E}(z)=\frac{1}{4\pi\epsilon_0}\int_{0}^{\pi}\frac{(2\pi\sigma\,R^2\sin{\theta}\,d\theta)(z-R\cos{\theta})}{((R\sin{\theta})^2+(z-R\cos{\theta})^2)^{3/2}}\hat{z}$$
$$u=((R\sin{\theta})^2+(z-R\cos{\theta})^2)^{-1/2}=(R^2+z^2-2zR\cos{\theta})^{-1/2}$$
$$du=-\frac{1}{2}\frac{2zR\sin{\theta}d{\theta}}{((R\sin{\theta})^2+(z-R\cos{\theta})^2)^{3/2}}$$
Limits of integration are $u=\frac{1}{z-R}$ when $\theta=0$ and $u=\frac{1}{z+R}$ when $\theta=\pi$
$$\vec{E}(z)=\frac{2\pi\sigma\,R}{4\pi\epsilon_0}\int_{0}^{\pi}\frac{(R\sin{\theta}\,d\theta)(z-R\cos{\theta})}{((R\sin{\theta})^2+(z-R\cos{\theta})^2)^{3/2}}\hat{z}=
$$
$$\vec{E}(z)=-\frac{2\pi\sigma\,R}{4\pi\epsilon_0\,z}\int_{\frac{1}{z-R}}^{\frac{1}{z+R}}(z-R\cos{\theta})du\,\hat{z}$$
Since $$z-R\cos{\theta}=z-R\left(\frac{R^2+z^2-\frac{1}{u^2}}{2zR} \right )=\left(\frac{2z^2-R^2-z^2+\frac{1}{u^2}}{2z} \right )z-R\cos{\theta}=\frac{z^2-R^2+\frac{1}{u^2}}{2z}$$
$$\vec{E}(z)=-\frac{2\pi\sigma\,R}{4\pi\epsilon_0\,z}\int_{\frac{1}{z-R}}^{\frac{1}{z+R}}\left(\frac{z^2-R^2+\frac{1}{u^2}}{2z} \right )du\,\hat{z}$$
$$\vec{E}(z)=-\frac{2\pi\sigma\,R}{4\pi\epsilon_0\,z}\left[\int_{\frac{1}{z-R}}^{\frac{1}{z+R}}\left(\frac{z^2-R^2}{2z} \right )du+\int_{\frac{1}{z-R}}^{\frac{1}{z+R}}\left(\frac{1}{2u^2z} \right )du\right ]\,\hat{z}
$$
$$\vec{E}(z)=-\frac{2\pi\sigma\,R}{4\pi\epsilon_0\,z}\left[\left(\frac{z^2-R^2}{2z} \right )\left(\frac{1}{z+R}-\frac{1}{z-R} \right )+\frac{1}{2z}\left.\left(-\frac{1}{u} \right )\right|_{\frac{1}{z-R}}^{\frac{1}{z+R}}\right ]\,\hat{z}$$
$$\vec{E}(z)=\left[\frac{4\pi{R^2}\sigma}{4\pi\epsilon_0\,z^2} \right ]\,\hat{z}$$
\subsection*{case( $z<R$)}
The location of point is inside the sphere.
To parametrize the shell, consider a radius vector making an angle $\theta$ with $Z$ axis. All the points that satisfy this condition are on a circle of radius $R\sin{\theta}$. Consider another radius vector making an angle $\theta+d\theta$ with $Z$ axis. All the points that satisfy this condition are on a circle of radius $R\sin{(\theta+d\theta)} $. The element is the area of sphere between these circles.The width of the element is $Rd\theta$. Area of the element is $2\pi(R\sin{\theta})(Rd\theta)=2\pi\,R^2\sin{\theta}\,d\theta$. Charge on the element is $dQ=2\pi\sigma\,R^2\sin{\theta}\,d\theta$
Electric field due to the charge $Q$ spread uniformly on a ring of radius $R$ at a distance $z$ along the axis of the ring is $\vec{E}(z)=\frac{1}{4\pi\epsilon_0}\frac{Qz}{(R^2+z^2)^{3/2}}\hat{z}$.
Using the above result,
Electric field due to the above element is $$d\vec{E}(z)=\frac{1}{4\pi\epsilon_0}\frac{(2\pi\sigma\,R^2\sin{\theta}\,d\theta)(z-R\cos{\theta})}{((R\sin{\theta})^2+(z-R\cos{\theta})^2)^{3/2}}\hat{z}$$
Integrating over the above result from $\theta=0$ to $\theta=\pi$
$$\vec{E}(z)=\int\,d\vec{E}(z)=\int_{0}^{\pi}\frac{1}{4\pi\epsilon_0}\frac{(2\pi\sigma\,R^2\sin{\theta}\,d\theta)(z-R\cos{\theta})}{((R\sin{\theta})^2+(z-R\cos{\theta})^2)^{3/2}}\hat{z}$$
$$\vec{E}(z)=\frac{1}{4\pi\epsilon_0}\int_{0}^{\pi}\frac{(2\pi\sigma\,R^2\sin{\theta}\,d\theta)(z-R\cos{\theta})}{((R\sin{\theta})^2+(z-R\cos{\theta})^2)^{3/2}}\hat{z}$$
$$u=((R\sin{\theta})^2+(z-R\cos{\theta})^2)^{-1/2}=(R^2+z^2-2zR\cos{\theta})^{-1/2}$$
$$du=-\frac{1}{2}\frac{2zR\sin{\theta}d{\theta}}{((R\sin{\theta})^2+(z-R\cos{\theta})^2)^{3/2}}$$
Limits of integration are $u=\frac{1}{R-z}$ when $\theta=0$ and $u=\frac{1}{R+z}$ when $\theta=\pi$.
$$\vec{E}(z)=\frac{2\pi\sigma\,R}{4\pi\epsilon_0}\int_{0}^{\pi}\frac{(R\sin{\theta}\,d\theta)(z-R\cos{\theta})}{((R\sin{\theta})^2+(z-R\cos{\theta})^2)^{3/2}}\hat{z}=-\frac{2\pi\sigma\,R}{4\pi\epsilon_0\,z}\int_{\frac{1}{R-z}}^{\frac{1}{R+z}}(z-R\cos{\theta})du\,\hat{z}$$
Since $$z-R\cos{\theta}=z-R\left(\frac{R^2+z^2-\frac{1}{u^2}}{2zR} \right )=\left(\frac{2z^2-R^2-z^2+\frac{1}{u^2}}{2z} \right )z-R\cos{\theta}=\frac{z^2-R^2+\frac{1}{u^2}}{2z}$$
$$\vec{E}(z)=-\frac{2\pi\sigma\,R}{4\pi\epsilon_0\,z}\int_{\frac{1}{R-z}}^{\frac{1}{R+z}}\left(\frac{z^2-R^2+\frac{1}{u^2}}{2z} \right )du\,\hat{z}$$
$$\vec{E}(z)=-\frac{2\pi\sigma\,R}{4\pi\epsilon_0\,z}\left[\int_{\frac{1}{R-z}}^{\frac{1}{R+z}}\left(\frac{z^2-R^2}{2z} \right )du+\int_{\frac{1}{R-z}}^{\frac{1}{R+z}}\left(\frac{1}{2u^2z} \right )du\right ]\,\hat{z}$$
$$\vec{E}(z)=-\frac{2\pi\sigma\,R}{4\pi\epsilon_0\,z}\left[\left(\frac{z^2-R^2}{2z} \right )\left(\frac{1}{R+z}-\frac{1}{R-z} \right )+\frac{1}{2z}\left.\left(-\frac{1}{u} \right )\right|_{\frac{1}{R-z}}^{\frac{1}{R+z}}\right ]\,\hat{z}$$
$$\vec{E}(z)=0$$


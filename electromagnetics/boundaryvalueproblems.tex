% !TeX root = main.tex
\section*{boundaryvalueproblems}
Laplace's equation in spherical polar coordinates is 
\[\frac{1}{r^2}\pdv{r}\qty(r^2\pdv{V}{r})+\frac{1}{r^2\sin\theta}\pdv{\theta}\qty(\sin\theta\pdv{V}{\theta})+\frac{1}{r^2\sin^2\theta}\pdv[2]{V}{\phi}=0\]
Since the problem is symmetric about the azimuthal angle $\phi$, \[\pdv{V}{\phi}=0\qq{and}\pdv[2]{V}{\phi}=0\]
Hence the Laplace's equation is 
\[\frac{1}{r^2}\pdv{r}\qty(r^2\pdv{V}{r})+\frac{1}{r^2\sin\theta}\pdv{\theta}\qty(\sin\theta\pdv{V}{\theta})=0\]
Substitute in the above equation and divide with \[V(r,\theta)=R(r)\Theta(\theta)\]  
\[\frac{1}{R}\pdv{r}\qty(r^2\pdv{R}{r})+\frac{1}{\Theta\sin\theta}\pdv{\theta}\qty(\sin\theta\pdv{\Theta}{\theta})=0\]
In the above expression, first term is independent of $\theta$ and second expression is independent of $r$. Hence each term is constant.
\[\dv{r}\qty(r^2\dv{R}{r})=l(l+1)R\]
\[\dv{\theta}\qty(\sin\theta\dv{\Theta}{\theta})=-l(l+1)\sin\theta\Theta\]
General solution of $r$-equation is 
\[R(r)=A_lr^l+\frac{B_l}{r^{l+1}}\]
Solutions of $\theta$-equation are Legendre polynomials $P_l(\cos\theta)$. 
General solution is \[V(r,\theta)=\sum_{l=0}^{\infty}\qty(A_lr^l+\frac{B_l}{r^{l+1}})P_l(\cos\theta)\]
\subsection*{spherical capacitor}
Potential at $r=R_1$ is $V_1$ and potential at $r=R_2$ is $V_2$.\\
In the region $r<R_1$, coefficients $B_{l}$ are zero as the solution may blow up at $r=0$.
Applying the boundary condition $V(r=R_1)=V_1$,
\[V(r,\theta)=\sum_{l=0}^{\infty}A_{l}R_1^lP_l(\cos\theta)=V_1\]
Mulitply by $P_m(\cos\theta)\sin\theta$ and integrating from $0$ to $\pi$,
\[A_lR_1^l\frac{2}{2l+1}=\int_0^{\pi}V_1P_l(\cos\theta)\sin\theta\dd\theta\]
Only $l=0$ term will survive. Remaining $A_l$ are zero.
\[A_0=V_1\]
Potential in the region $r<R_1$ is \[V(r,\theta)=V_1\]
\newpage
In the region $r>R_2$, coefficients $A_{l}$ are zero as the solution may blow up at $r=\infty$.
Applying the boundary condition $V(r=R_2)=V_2$,
\[V(r,\theta)=\sum_{l=0}^{\infty}\frac{B_{l}}{R_2^{l+1}}P_l(\cos\theta)=V_2\]
Mulitply by $P_m(\cos\theta)\sin\theta$ and integrating from $0$ to $\pi$,
\[\frac{B_l}{R_2^{l+1}}\frac{2}{2l+1}=\int_0^{\pi}V_2P_l(\cos\theta)\sin\theta\dd\theta\]
Only $l=0$ term will survive. Remaining $B_l$ are zero.
\[\frac{B_0}{R_2}=V_2\]
Potential in the region $r>R_2$ is \[V(r,\theta)=\frac{V_2R_2}{r}\]
In the region $R_1<r<R_2$, both set of coefficients are nonzero.
Using the boundary conditions,
\[V(r,\theta)=\sum_{l=0}^{\infty}\qty(A_{l}R_2^l+\frac{B_{l}}{R_2^{l+1}})P_l(\cos\theta)=V_2\]
\[V(r,\theta)=\sum_{l=0}^{\infty}\qty(A_{l}R_1^l+\frac{B_{l}}{R_1^{l+1}})P_l(\cos\theta)=V_1\]
Multiply by $P_m(\cos\theta)\sin\theta$ and integrate from $0$ to $\pi$,
\[\qty(A_{l}R_2^l+\frac{B_{l}}{R_2^{l+1}})\frac{2}{2l+1}=\int_0^{\pi}V_2P_l(\cos\theta)\sin\theta\dd\theta\]
only the $l=0$ term will survive.
\[A_{0}+\frac{B_{0}}{R_2}=V_2\]
\[A_{0}+\frac{B_{0}}{R_1}=V_1\]
\[B_0=\frac{V_2-V_1}{\frac{1}{R_2}-\frac{1}{R_1}}\]
\[A_0=\frac{V_2R_2-V_1R_1}{R_2-R_1}\]
\[V(r,\theta)=A_0+\frac{B_0}{r}=\frac{V_2R_2-V_1R_1}{R_2-R_1}-\frac{(V_2-V_1)R_1R_2}{r(R_2-R_1)}\]
\subsection*{(1)}
Two concentric spheres have radii $a$ and $b$. If the potential of inner sphere is $V_aP_3(\cos\theta)$ and the potential of the outer sphere is $V_bP_5(\cos\theta)$, find the potential for all the points between $a<r<b$.


% !TEX TS-program = pdflatex
% !TEX encoding = UTF-8 Unicode

\documentclass[12pt]{article}

\usepackage[utf8]{inputenc} 

\usepackage{geometry}
\geometry{a4paper} 
\geometry{margin=0.25in} 
\geometry{portrait}

\usepackage{amsmath}
\usepackage{physics}
\usepackage{tikz} 

\title{polarized sphere}
\author{vijayabhaskar badireddi}
%\date{} 

\begin{document}
%\maketitle
\section*{polarized sphere}
Consider a sphere of radius $R$, with uniform polarization $\vb{P}$.\\Volume bound charge density $\rho_b=-\div{\vb{P}}=0$ as $\vb{P}$ is uniform.\\ Surface bound charge density $\sigma_b=\vb{P}\cdot\vu{n}$. Let $\vb{P}$ be in the direction of $-z$ axis. $\vb{P}=-P\vu{z}$. $\vu{n}$ is outward unit vector normal to the surface. That is $\vu{n}=\vu{r}$ where $\vu{r}$ is unit vector along $r$ for spherical coordinates. $\sigma_b=\vu{n}\cdot\vu{r}=-P\vu{z}\cdot\vu{r}$.\[\vu{r}=\sin\theta\cos\varphi\vu{\theta}+\sin\theta\sin\varphi\vu{\varphi}+\cos\theta\vu{z}\]
\[\sigma_b=\vu{n}\cdot\vu{r}=-P\vu{z}\cdot\vu{r}=-P\cos\theta\]
Comparing this with the given charge density $\sigma=\sigma_0\cos\theta$, $\sigma_0=-P$ and $\sigma=-P\cos\theta$.\\
Since the charge densities are identical, electric fields inside and outside the sphere are also identical in both cases.\\
\subsection*{Electric field inside and outside the sphere}
Due to spherical symmetry of the problem, the potential $V(r,\theta)$ due to sphere inside and outside can be expressed in terms of Legendre polynomials.

\[V(r,\theta)=\sum_{l=1}^{\infty}A_lr^lP_l(\cos\theta)\quad(r\leq R)\]
\[V(r,\theta)=\sum_{l=1}^{\infty}B_lr^{-(l+1)}P_l(\cos\theta)\quad(r\geq R)\]
Since the potential is continuous at $r=R$, equating the above expressions $B_l=A_lR^{2l+1}$.\\
Normal derivative of $V(r,\theta)$ at $r=R$ has a discontinuity.
\[\qty(\pdv{V}{n})_{above}-\qty(\pdv{V}{n})_{below}=-\frac{\sigma}{\epsilon_0}\]
\[-\sum_{l=1}^{\infty}(l+1)B_lr^{-(l+2)}P_l(\cos\theta)-\sum_{l=1}^{\infty}lA_lr^{l-1}P_l(\cos\theta)=-\frac{\sigma_0\cos\theta}{\epsilon_0}\]
\[A_l=\frac{1}{2\epsilon_0R^{l-1}}\int_0^{\pi}(\sigma_0\cos\theta)P_l(\cos\theta)\sin\theta\dd\theta\]
All the $A_l$ are zero except for $l=1$. $P_l(\cos\theta)=\cos\theta$.
\[A_1=\frac{\sigma_0}{2\epsilon_0}\int_0^{\pi}\cos^2\theta\sin\theta\dd\theta=\frac{\sigma_0}{3\epsilon_0}\]
Hence potential inside the sphere is \[V(r,\theta)=\frac{\sigma_0}{3\epsilon_0}r\cos\theta\quad(r\leq R)\]
\[V(r,\theta)=\frac{\sigma_0R^3}{3\epsilon_0}\frac{\cos\theta}{r^2}\quad(r\geq R)\]
Electric field inside the sphere is \[\vb{E}=-\grad{V}=\frac{\sigma_0}{3\epsilon_0}\qty[\pdv{r}(r\cos\theta)\vu{r}+\pdv{\theta}(r\cos\theta)\vu{\theta}+\frac{1}{r\cos\theta}\pdv{\varphi}(r\cos\theta)\vu{\varphi}]\]
\[=-\frac{\sigma_0}{3\epsilon_0}\qty[\cos\theta\vu{r}-\sin\theta\vu{\theta}]=-\frac{\sigma_0}{3\epsilon_0}\vu{z}\]
 \[\qq{Since $\sigma_0=-P$,}\vb{E}=\frac{P}{3\epsilon_0}\vu{z}=-\frac{\vb{P}}{3\epsilon_0}\]
 Electric field outside the sphere is \[\vb{E}=-\grad{V}=-\frac{\sigma_0R^3}{3\epsilon_0}\qty[\pdv{r}\qty(\frac{\cos\theta}{r^2})\vu{r}+\frac{1}{r}\pdv{\theta}\qty(\frac{\cos\theta}{r^2})\vu{\theta}+\frac{1}{r\sin\theta}\pdv{\varphi}\qty(\frac{\cos\theta}{r^2})\vu{\varphi}]\]
\[=-\frac{\sigma_0R^3}{3\epsilon_0}\qty[-\frac{2\cos\theta}{r^3}\vu{r}-\frac{\sin\theta}{r^3}\vu{\theta}]\]
Potential outside the sphere is that of a point electric dipole at the origin with dipole moment \[\vb{p}=\frac{4\pi R^3}{3}\vb{P}=-\frac{4\pi R^3P}{3}\vb{z}=\frac{4\pi R^3\sigma_0}{3}\vb{z}\]
\[V(r,\theta)=\frac{1}{4\pi\epsilon_0}\frac{\vb{p}\cdot\vu{r}}{r^2}=\frac{1}{4\pi\epsilon_0}\qty(\frac{\sigma_0\vu{z}\cdot\vu{r}}{r^2})\qty(\frac{4\pi R^3}{3})=\frac{\sigma_0R^3}{3\epsilon_0}\frac{\cos\theta}{r^2}\]
Hence electric field outside the sphere si also that of a point dipole located at origin, with dipole moment $\vb{p}=\frac{4\pi R^3}{\epsilon_0}\vb{P}$.
\[\vb{E}=\frac{\sigma_0 R^3}{3\epsilon_0}\qty[\frac{2\cos\theta}{r^3}\vu{r}+\frac{\sin\theta}{r^3}\vu{\theta}]\]

\end{document}

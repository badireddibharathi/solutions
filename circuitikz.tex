% !TEX TS-program = pdflatex
% !TEX encoding = UTF-8 Unicode

\documentclass[12pt]{article}

\usepackage[utf8]{inputenc} 

\usepackage{geometry}
\geometry{a4paper}
\geometry{margin=0.25in} 
\geometry{portrait}

\usepackage{amsmath}
\usepackage{physics}
\usepackage{siunitx}
\usepackage{circuitikz}

\title{circuitikz figures}
\author{vijayabhaskar badireddi}

\begin{document}

\section*{circuitikz figures}
\subsection*{figure 1}
\begin{center}
\begin{circuitikz}[scale=1]
 \draw (0,0) to[R,l=$R_1$,f=$i_1$] (4,0) to [nos,l=$S$] (4,-2) to [battery1,l=$\mathcal{E}$] (4,-4) to[C,l=$C_1$] (0,-4) to [L,l=$L_1$,i=$i_1$] (0,0);
 \end{circuitikz}
\end{center}

\subsection*{figure 2}
\begin{center}
\begin{circuitikz}[scale=1]
 \draw (0,0) 
 node [anchor=east] {a}
 to[battery1,l_=$\mathcal{E}$,*-] (2,0) to [nos,l_=$S$,-*] (4,0)
  node [anchor=west] {b} ;
 \draw (0,0) to [R,l=$R_1$,f=$i_1$] (0,3);
 \draw (4,0) to
 node [anchor=south,pos=2.5] {$+q$} (4,0) 
 to [C,l_=$C_1$] (4,3);
 \draw (0,3) to [L,l=$L_1$,i=$i_1$] (4,3);
 \end{circuitikz}
\end{center}

\begin{center}
\begin{circuitikz}
  \draw (0,0) 
  	node [anchor=east]{A}
        to node[above,pos=2]{$+q$} (0.5,0) 
        to [C, l_=$C$] (2.5,0) 
        to node[above,pos=-1]{$-q$} (3,0)
        node [anchor=west] {B} ;
\end{circuitikz}

\end{center}
\end{document}

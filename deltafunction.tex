% !TEX TS-program = pdflatex
% !TEX encoding = UTF-8 Unicode
\documentclass[12pt]{article} 

\usepackage[utf8]{inputenc} 
\usepackage{geometry} 
\geometry{a4paper} 
\geometry{margin=0.25in} 
\geometry{portrait} 

\usepackage{tikz} 

\usepackage{amsmath} 
\usepackage{physics} 
\title{deltafunction}
\author{vijayabhaskar badireddi} 
\date{} 

\begin{document}

\section*{delta function}
\subsection*{spherical coordinates}
\[\qq{line element}\dd{\vb{l}}=\dd{r}\vu{r}+r\dd{\theta}\vu{\theta}+r\sin\theta\dd{\phi}\vu{\phi}\]

\[\delta(\vb{r}-\vb{r}_0)=\frac{1}{r^2\sin\theta}\delta(r-r_0)\delta(\theta-\theta_0)\delta(\phi-\phi_0)\]
\[\int f(\vb{r})\delta(\vb{r}-\vb{r}_0)\dd^3{r}=Q\]
\[\int f(\vb{r})\frac{1}{r^2\sin\theta}\delta(r-r_0)\delta(\theta-\theta_0)\delta(\phi-\phi_0)\dd^3{r}=Q\]
\[\dd^3{r}=r^2\sin\theta\dd{r}\dd{\theta}\dd{\phi}\]
\[\qq{azimuthal symmetry} \delta(\vb{r}-\vb{r}_0)=\frac{1}{2\pi r^2\sin\theta}\delta(r-r_0)\delta(\theta-\theta_0)\]
\[\qq{spherical symmetry} \delta(\vb{r}-\vb{r}_0)=\frac{1}{4\pi r^2}\delta(r-r_0)\]
\[\qq{cylindrical coordinates}\delta(\vb{r}-\vb{r}_0)=\frac{1}{s}\dd{s}\dd{\phi}\dd{z}\]
\[\delta(\vb{r}-\vb{r}_0)=\frac{1}{s}\delta{(s-s_0)}\delta{(\phi-\phi_0)}\delta{(z-z_0)}\]
\subsection*{point }
\[\laplacian\qty(\frac{1}{|\vb{r}-\vb{r}'|})=-4\pi\delta(\vb{r}-\vb{r}')\]
\[\int_V\laplacian\qty(\frac{1}{|\vb{r}-\vb{r}'|})\dd[3]{r}=\int_V\div\grad\qty(\frac{1}{|\vb{r}-\vb{r}'|})\dd[3]{r}=\int_S\grad\qty(\frac{1}{|\vb{r}-\vb{r}'|})\cdot\dd{S}=-4\pi\]
\subsection*{charge densities}
A charge $Q$ is distributed over a spherical surface of radius $R$. In spherical coordinates find the volume charge density.
\[\rho(\vb{r})=\frac{Q}{4\pi R^2}\delta(r-R)\] 
Charge with Linear charge density $\lambda$ is distributed on a hollow cylindrical surface of radius $b$. Find volume charge density \[\rho(\vb{r})=\frac{\lambda}{2\pi b}\delta(r-b)\]
Charge $Q$ is distributed uniformly on a disk of radius $R$. Find volume charge density.\[\frac{Q}{\pi R^2}\delta(z)\Theta(R-r)\]
Charge $Q$ is distributed uniformly on a disk of radius $R$. Find volume charge density.\[\frac{Q}{\pi R^2r}\delta\qty(\theta-\frac{\pi}{2})\Theta(R-r)\]
\end{document}

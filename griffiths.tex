% !TEX TS-program = pdflatex
% !TEX encoding = UTF-8 Unicode

\documentclass[12pt]{article}

\usepackage[utf8]{inputenc} 

\usepackage{geometry}
\geometry{a4paper} 
\geometry{margin=0.25in} 
\geometry{portrait}

\usepackage{amsmath}
\usepackage{physics}
\usepackage{tikz} 

\title{Electrodynamics}
\author{vijayabhaskar badireddi}
%\date{} 

\begin{document}
%\maketitle

\section{Vector Analysis}

\[\dd{T}=\pdv{T}{x}\dd{x}+\pdv{T}{y}\dd{y}+\pdv{T}{z}\dd{z}=\grad{T}\vdot\dd{\vb{l}}\]
The gradient $\grad{T}$ points in the direction of  maximum increase in the function $T$. The magnitude $|\grad{T}|$ gives the slope along the maximal direction.

\subsection*{Cartesian}
\[\dd{\vb{l}}=\dd{x}\vu{\vb{x}}+\dd{y}\vu{\vb{y}}+\dd{z}\vu{\vb{z}}\quad\dd{\tau}=\dd{x}\dd{y}\dd{z}\]
\[\qq{Gradient}\quad\grad{f}=\pdv{f}{x}\vu{\vb{x}}+\pdv{f}{y}\vu{\vb{y}}+\pdv{f}{z}\vu{\vb{z}}\]
\[\qq{Divergence}\quad\div{\vb{V}}=\pdv{V_x}{x}+\pdv{V_y}{y}+\pdv{V_z}{z}\]
\[\qq{Curl}\quad\curl{\vb{V}}=\qty(\pdv{V_z}{y}-\pdv{V_y}{z})\vu{\vb{x}}+\qty(\pdv{V_x}{z}-\pdv{V_z}{x})\vu{\vb{y}}+\qty(\pdv{V_y}{x}-\pdv{V_x}{y})\vu{\vb{z}}\]
\[\qq{Laplacian}\quad\laplacian{f}=\pdv[2]{f}{x}+\pdv[2]{f}{y}+\pdv[2]{f}{z}\]

\subsection*{Spherical coordinates}
\begin{align*}
\vu{r}&=\sin\theta\cos\phi\vu{x}+\sin\theta\cos\phi\vu{y}+\cos\theta\vu{z}\\
\vu{\theta}&=\cos\theta\cos\phi\vu{x}+\cos\theta\sin\phi\vu{y}-\sin\theta\vu{z}\\
\vu{\phi}&=-\sin\phi\vu{x}+\cos\phi\vu{y}
\end{align*}
\[\dd{\vb{l}}=\dd{r}\vu{\vb{r}}+r\dd{\theta}\vu{\vb{\theta}}+r\sin\phi\dd{\phi}\vu{\vb{\phi}}\quad\dd{\tau}=r^2\sin\theta\dd{r}\dd{\theta}\dd{\phi}\]
\[\qq{Gradient}\quad\grad{f}=\pdv{f}{r}\vu{\vb{r}}+\frac{1}{r}\pdv{f}{\theta}\vu{\vb{\theta}}+\frac{1}{r\sin\theta}\pdv{f}{\phi}\vu{\vb{\phi}}\]
\[\qq{Divergence}\quad\div{\vb{V}}=\frac{1}{r^2}\pdv{r}\qty(r^2V_r)+\frac{1}{r\sin\theta}\pdv{\theta}\qty(\sin{\theta}V_{\theta})+\frac{1}{r\sin\theta}\pdv{V_{\phi}}{\phi}\]
\[\qq{Curl}\quad\curl{\vb{V}}=\frac{1}{r\sin\theta}\bqty{\pdv{\theta}\qty(\sin{\theta}V_{\phi})-\pdv{V_{\theta}}{\phi}}\vu{\vb{r}}+\frac{1}{r}\bqty{\frac{1}{\sin\theta}\pdv{V_r}{\phi}-\pdv{r}\qty(rV_{\phi})}\vu{\vb{\theta}}+\frac{1}{r}\bqty{\pdv{r}\qty(rV_{\theta})-\pdv{V_r}{\theta}}\vu{\vb{\phi}}\]
\[\qq{Laplacian}\quad\laplacian{f}=\frac{1}{r^2}\pdv{r}\qty(r^2\pdv{f}{r})+\frac{1}{r^2\sin\theta}\pdv{\theta}\qty(\sin{\theta}\pdv{f}{\theta})+\frac{1}{r^2\sin^2{\theta}}\pdv[2]{f}{\phi}\]

\subsection*{Cylindrical coordinates}
\begin{align*}
\vu{s}&=\cos\phi\vu{x}+\sin\phi\vu{y}\\
\vu{\phi}&=-\sin\phi\vu{x}+\cos\phi\vu{y}\\
\vu{z}&=\vu{z}
\end{align*}
\[\dd{\vb{l}}=\dd{s}\vu{\vb{s}}+s\dd{\phi}\vu{\vb{\phi}}+\dd{z}\vu{\vb{z}}\quad\dd{\tau}=s\dd{s}\dd{\phi}\dd{z}\]
\[\qq{Gradient}\quad\grad{f}=\pdv{f}{s}\vu{\vb{s}}+\frac{1}{s}\pdv{f}{\phi}\vu{\vb{\phi}}+\pdv{f}{z}\vu{\vb{z}}\]
\[\qq{Divergence}\quad\div{\vb{V}}=\frac{1}{s}\pdv{s}\qty(sV_s)+\frac{1}{s}\pdv{V_{\phi}}{\phi}+\pdv{V_z}{z}\]
\[\qq{Curl}\quad\curl{\vb{V}}=\bqty{\frac{1}{s}\pdv{V_z}{\phi}-\pdv{V_{\phi}}{z}}\vu{\vb{s}}+\bqty{\pdv{V_s}{z}-\pdv{V_z}{s}}\vu{\vb{\phi}}+\frac{1}{s}\bqty{\pdv{s}\qty(sV_{\phi})-\pdv{V_s}{\phi}}\vu{\vb{z}}\]
\[\qq{Laplacian}\quad\laplacian{f}=\frac{1}{s}\pdv{s}\qty(s\pdv{f}{s})+\frac{1}{s^2}\pdv[2]{f}{\phi}+\pdv[2]{f}{z}\]
\subsection*{vector identities}
\[\vb{A}\vdot\qty(\vb{B}\cp\vb{C})=\vb{B}\vdot\qty(\vb{C}\cp\vb{A})=\vb{C}\vdot\qty(\vb{A}\cp\vb{B})\]
\[\vb{A}\cp\qty(\vb{B}\cp\vb{C})=\vb{B}\qty(\vb{A}\vdot\vb{C})-\vb{C}\qty(\vb{A}\vdot\vb{B})\]
\[\grad{(fg)}=f(\grad{g})+g(\grad{f})\]
\[\grad{(\vb{A}\vdot\vb{B})}=\vb{A}\cp\qty(\curl{\vb{B}})+\vb{B}\cp\qty(\curl{\vb{A}})+\qty(\vb{A}\vdot\nabla)\vb{B}+\qty(\vb{B}\vdot\nabla)\vb{A}\]
\[\div{\qty(f\vb{A})}=f(\div{\vb{A}})+\vb{A}\vdot\qty(\grad{f})\]
\[\div{\qty(\vb{A}\cp\vb{B})}=\vb{B}\vdot\qty(\curl{\vb{A}})-\vb{A}\vdot\qty(\curl{\vb{B}})\]
\[\curl{\qty(f\vb{A})}=f(\curl{\vb{A}})-\vb{A}\cp\qty(\grad{f})\]
\[\curl{(\vb{A}\cp\vb{B})}=\qty(\vb{B}\vdot\nabla)\vb{A}-\qty(\vb{A}\vdot\nabla)\vb{B}+\vb{A}\qty(\div{\vb{B}})-\vb{B}\qty(\div{\vb{A}})\]
\[\div{\curl{\vb{A}}}=0\quad\curl{\grad{f}}=0\quad\curl{\curl{\vb{A}}}=\grad\qty{\curl{\vb{A}}}-\laplacian{\vb{A}}\]
\[\qq{gradient theorem}\quad \int_{\vb{a}}^{\vb{b}}\qty(\grad{f})\vdot\dd{\vb{l}}=f(\vb{b})-f(\vb{a})\]
\[\qq{divergence theorem} \int\qty(\div{\vb{A}})\dd{\tau}=\oint\vb{A}\vdot\dd{\vb{a}}\]
\[\qq{curl theorem} \int\qty(\curl{\vb{A}})\vdot\dd{\vb{a}}=\oint\vb{A}\vdot\dd{\vb{l}}\]

\subsection*{Dirac delta function}
\[\div{\qty(\frac{\vb{r}}{r^2})}=4\pi\delta^3(\vb{r})\]
\[\laplacian{\qty(\frac{1}{r})}=-4\pi\delta^3(\vb{r})\]
\section{Electrostatics}
\[\qq{Coulomb's law}\vb{F}_{12}=\frac{1}{4\pi\epsilon_0}\frac{q_1q_2(\vb{r}_1-\vb{r}_2)}{\vqty{\vb{r}_1-\vb{r}_2}^3}\]
\[\qq{Gauss law}\quad \oint\vb{E}\vdot\dd{\vb{a}}=\frac{Q_{enc}}{\epsilon_0}\]
\[\div{\vb{E}}=\frac{\rho}{\epsilon_0}\]
\[\div{\qty(\frac{\vb{r}-\vb{r}'}{\vqty{\vb{r}-\vb{r}'}^2})}=4\pi\delta^3(\vb{r}-\vb{r}')\]
\[\curl{\vb{E}}=0\quad \oint\vb{E}\vdot\dd{\vb{l}}=0\]
\[V(\vb{r})=-\int_{\infty}^{\vb{r}}\vb{E}\vdot\dd{\vb{l}}\quad\vb{E}=-\grad{V}\]
\[\qq{Poisson's equation}\quad\laplacian{V}=-\frac{\rho}{\epsilon_0}\]
\[\qq{Laplace's equation} \quad\laplacian{V}=0\]
\[\qq{boundary conditions} \vb{E}^{\parallel}_{above}=\vb{E}^{\parallel}_{below}\quad \vb{E}_{above}-\vb{E}_{below}=\frac{\sigma}{\epsilon_0}\vu{n} \]
\[W=\frac{1}{2}\int\rho{V}\dd{\tau}=\frac{\epsilon_0}{2}\int{E^2}\dd\tau\]

\section{Special techniques}
\[\qq{First uniqueness theorem}\]
The solution to Laplace's equation in some volume is uniquely determined by specifying potential on the boundary surface.
\[\qq{Second uniqueness theorem}\]
In a volume surrounded by conductors and containing a specified charge density, the electric field is uniquely determined if the total charge on each conductor is given.

\subsection*{method of images}

\subsection*{separation of variables}

\subsection*{multipole expansion}

\section{Electric fields in matter}
\[\vb{N}=\vb{p}\cp\vb{E}\quad\vb{F}=\qty(\vb{p}\vdot\nabla)\vb{E}\quad U=-\vb{p}\vdot\vb{E}\]
\subsection*{polarization}
\[\sigma_b=\vb{P}\vdot\vu{n}\quad\rho_b=-\div{\vb{P}}\]
\subsection*{displacement}
\subsection*{misc}

\subsection*{Coulomb's law}
\[\vb{E}=\frac{1}{4\pi\epsilon_0}\frac{q_1q_2(\vb{r}_1-\vb{r}_2)}{|\vb{r}_1-\vb{r}_2|^3}\]
\[V(\vb{r})=-\int_\infty^{\vb{r}}\vb{E}\cdot\dd\vb{r}\]
\[\vb{E}=-\grad{V}\]

\subsection*{Gauss' law}
\[\oint_S{\vb{E}\cdot\dd{\vb{a}}}=\frac{Q_{enc}}{\epsilon_0}\]
\[\laplacian{V}=-\frac{\rho}{\epsilon_0}\]
\[\pdv[2]{V}{x}+\pdv[2]{V}{y}+\pdv[2]{V}{z}=0\]
\[\frac{1}{r^2}\pdv{r}\qty(r\pdv{V}{r})+\frac{1}{r\sin\theta}\pdv{\theta}\qty(\sin\theta\pdv{V}{\theta})+\frac{1}{r^2\sin^2\theta}\pdv[2]{V}{\phi}=0\]

\[\int_{v}(\div{\vb{A}})\dd\tau=\oint_{s}\vb{A}\cdot\dd{\vb{a}}\]
\[\int_{v}(\div{\vb{A}})\dd\tau=\oint_{s}\vb{A}\cdot\dd{\vb{a}}\]
\[\grad{(\vb{a}\cdot\vb{b})}=\vb{a}\times(\curl{\vb{b}})+\vb{b}\times(\curl{\vb{a}})+(\vb{a}\cdot\nabla)\vb{b}+(\vb{b}\cdot\nabla)\vb{a}\]
\[\div{(f\vb{a})}=\vb{a}\cdot\grad{f}+f\div{\vb{a}}\]
\[\div{(\vb{a}\times\vb{b})}=\vb{b}\cdot(\curl{\vb{a}})-\vb{a}\cdot(\curl{\vb{b}})\]

\section{Magnetostatics}
\[\qq{Lorenz force law}\vb{F}=q(\vb{E}+\vb{v}\cp\vb{B})\]
\[\qq{Lorenz force law}\vb{F}=\int I(\dd\vb{l}\cp\vb{B})\]
\[\vb{F}=\int (\vb{K}\cp\vb{B})\dd{a}=\int (\sigma\vb{v}\cp\vb{B})\dd{a}\]
\[\vb{F}=\int (\vb{J}\cp\vb{B})\dd{v}=\int (\rho\vb{v}\cp\vb{B})\dd{v}\]
\[\vb{B}(\vb{r})=\frac{\mu_0}{4\pi}\int\frac{\vb{I}\cp(\vb{r}-\vb{r}')}{|\vb{r}-\vb{r}'|^2}\dd{l'}=\frac{\mu_0I}{4\pi}\int\frac{\dd\vb{l'}\cp(\vb{r}-\vb{r}')}{|\vb{r}-\vb{r}'|^2}\]
\[\curl{\vb{B}}=\mu_0\vb{J}\]
\[\oint\vb{B}\vdot\dd{\vb{l}}=\mu_0I_{enc}\]
\[\div{\vb{B}}=0\]
\[\vb{B}=\curl{\vb{A}}\]
\[\curl{\vb{A}}=0\]
\[\laplacian{\vb{A}}=-\mu_0\vb{J}\]
\[\vb{A(\vb{r})}=\frac{\mu_0}{4\pi}\int\frac{\vb{J(\vb{r'})}}{|\vb{r}-\vb{r'}|}\dd{\tau'}=\frac{\mu_0I}{4\pi}\int\frac{\dd{\vb{l}'}}{|\vb{r}-\vb{r'}|}=\frac{\mu_0}{4\pi}\int\frac{\vb{K}}{|\vb{r}-\vb{r'}|}\dd{a'}\]
\subsection*{boundary conditions}
\[B^{\perp}_{above}-B^{\perp}_{below}=0\]
\[B^{\parallel}_{above}-B^{\parallel}_{below}=\mu_oK\]
\[\vb{B}_{above}-\vb{B}_{below}=\mu_0(\vb{K}\cp\vu{n})\]
\[\vb{A}_{above}-\vb{A}_{below}=0\]
\[\pdv{\vb{A}_{above}}{n}-\pdv{\vb{A}_{below}}{n}=-\mu_0\vb{K}\]
multipole expansion of vector potential\[\vb{A(\vb{r})}=\frac{\mu_0I}{4\pi}\int\frac{\dd{\vb{l}'}}{|\vb{r}-\vb{r'}|}=\frac{\mu_0I}{4\pi}\sum_{n=0}^{\infty}\frac{1}{r^{n+1}}\oint(r')^nP_n(\cos\theta')\dd{\vb{l'}}\]

\section{Magnetic fields in matter}
Torque on a magnetic dipole $\vb{N}=\vb{m}\cp\vb{B}$.\\
Force on a magnetic dipole $\vb{F}=\grad(\vb{m}\vdot\vb{B})$.\\
Magnetic dipole moment per unit volume $\vb{M}$.\\
Vector potential $\vb{A}$ due to a magnetic dipole of dipole moment $\vb{m}$ \[\vb{A}(\vb{r})=\frac{\mu_0}{4\pi}\frac{\vb{m}\cp(\vb{r}-\vb{r}')}{|\vb{r}-\vb{r}'|^2}\]
\[\vb{A}(\vb{r})=\frac{\mu_0}{4\pi}\int\frac{\vb{M}(\vb{r'})\cp(\vb{r}-\vb{r}')}{|\vb{r}-\vb{r}'|^2}\dd{\tau'}\]
\[\vb{A}(\vb{r})=\frac{\mu_0}{4\pi}\left[\int\frac{\mathbf{\nabla'}\cp{\vb{M}(\vb{r'})}}{|\vb{r}-\vb{r}'|}\dd{\tau'}-\int\mathbf{\nabla'}\cp\left[{\frac{\vb{M}(\vb{r'})}{|\vb{r}-\vb{r}'|}}\right]\dd{\tau'}\right]\]

\[\vb{A}(\vb{r})=\frac{\mu_0}{4\pi}\left[\int\frac{\mathbf{\nabla'}\cp{\vb{M}(\vb{r'})}}{|\vb{r}-\vb{r}'|}\dd{\tau'}+\int\frac{[{\vb{M}(\vb{r'})}\cp\dd{\vb{a}'}]}{|\vb{r}-\vb{r}'|}\right]\]
Current density of bound current $\vb{J}_b=\curl{\vb{M}}$.Surface density of bound current $\vb{K}_b=\vb{M}\cp\vu{n}$.

\[\vb{A}(\vb{r})=\frac{\mu_0}{4\pi}\int_{\mathcal{V}}\frac{\vb{J}_b(\vb{r}')}{|\vb{r}-\vb{r}'|}\dd{\tau'}+\frac{\mu_0}{4\pi}\oint_{\mathcal{S}}\frac{\vb{K}_b(\vb{r}')}{|\vb{r}-\vb{r}'|}\dd{a}'\]
\[\vb{J}=\vb{J}_b+\vb{J}_f\]
\[\curl{\qty(\frac{1}{\mu_0}\vb{B}-\vb{M})}=\vb{J}_f\]
\[\vb{H}=\frac{1}{\mu_0}\vb{B}-\vb{M}\]
\[\curl{\vb{H}}=\vb{J}_f\]
\[\oint\vb{H}\vdot\dd{\vb{l}}=I_{f,enc}\]
\subsection*{Magneto static boundary conditions}
\[H_{above}^{\perp}-H_{below}^{\perp}=-(M_{above}^{\perp}-M_{above}^{\perp})\]
\[\vb{H}_{above}^{\parallel}-\vb{H}_{below}^{\parallel}=\vb{K}_f\cp\vb{n}\]
\subsection*{linear and nonlinear media}
\[\vb{M}=\chi_m\vb{H}\]
\[\vb{B}=\mu_0(\vb{H}+\vb{M})=\mu_0(1+\chi_m)\vb{H}=\mu\vb{H}\]
\section{Electrodynamics}
\[\qq{Ohm's law} \vb{J}=\sigma\vb{E}\]
\[\mathcal{E}=-\dv{\Phi}{t}\]
\[\curl{\vb{E}}=-\pdv{\vb{B}}{t}\]
\[\Phi_2=\int\vb{B}_1\vdot\dd{\vb{a}_2}\]
\[M_{21}=M_{21}=\frac{\mu_0I_1I_2}{4\pi}\oint\oint\frac{\dd{\vb{l}_1}\vdot\dd{\vb{l}_2}}{|\vb{r}-\vb{r}' |}\]
\[W=\frac{1}{2}LI^2=\frac{1}{2}\int B^2\dd{\tau}\]
\subsection*{Maxwell's laws}

\subsection*{Boundary conditions}
\[\qq{Gauss law for displacement}\oint_{S}\vb{D}\vdot\dd{\vb{a}}=Q_{free,enc}\]
\[\qq{no magnetic monopoles}\oint_{S}\vb{B}\vdot\dd{\vb{a}}=0\]
\[\oint\vb{E}\vdot\dd{\vb{l}}=-\dv{t}\int\vb{B}\vdot\dd{\vb{a}}\]
\[\oint\vb{H}\vdot\dd{\vb{l}}=I_{free,enc}+\dv{t}\int\vb{D}\vdot\dd{\vb{a}}\]
\[D_{1}^{\perp}-D_{2}^{\perp}=\epsilon_1E_{1}^{\perp}-\epsilon_2E_{2}^{\perp}=\sigma_f\]
\[B_{1}^{\perp}-B_{2}^{\perp}=0\]
\[\vb{E}_{1}^{\parallel}-\vb{E}_{2}^{\parallel}=0\]
\[\vb{H}_{1}^{\parallel}-\vb{H}_{2}^{\parallel}=\frac{1}{\mu_1}\vb{B}_{1}^{\parallel}-\frac{1}{\mu_2}\vb{B}_{2}^{\parallel}=\vb{K}_f\cp\vu{n}\]

\section{Conservation Laws}
\[\qq{Equation of continuity} \pdv{\rho}{t}=-\div{\vb{J}}\]
Poynting's theorem
\[\dv{W}{t}=\int_{\mathcal{V}}(\vb{E}\vdot{\vb{J}})\dd{\tau}=-\dv{t}\int_{\mathcal{V}}\frac{1}{2}\qty(\epsilon_0E^2+\frac{1}{\mu_0}B^2)\dd{\tau}-\frac{1}{\mu_0}\oint(\vb{E}\cp\vb{B})\vdot\dd{\vb{a}}\]
\[\qq{Poynting vector}\vb{S}=\frac{1}{\mu_0}(\vb{E}\cp\vb{B})\]
\[\qq{Maxwell's stress tensor} T_{ij}=\epsilon_0\qty(E_iE_j-\frac{1}{2}\delta_{ij}E^2)+\frac{1}{\mu_0}\qty(B_iB_j-\frac{1}{2}\delta_{ij}B^2)\]
\[\qq{force per unit volume} \vb{f}=\rho\vb{E}+\vb{J}\cp\vb{B}=\div{\overleftrightarrow{\vb{T}}}-\epsilon_0\mu_0\pdv{\vb{S}}{t}\]
\[\vb{F}=\oint_{\mathcal{S}}\overleftrightarrow{\vb{T}}\vdot\dd{\vb{a}}-\epsilon_0\mu_0\dv{t}\int_{\mathcal{V}}\vb{S}\dd{\tau}\]
\[\vb{p}_{em}=\mu_0\epsilon_0\int_{\mathcal{V}}\vb{S}\dd{\tau}\]
\[\qq{density of electromagnetic field momentum}\mathcal{P}_{em}=\mu_0\epsilon_0\vb{S}\]
\[\pdv{t}(\mathcal{P}_{mech}+\mathcal{P}_{em})=\div{\overleftrightarrow{\vb{T}}}\]
\section{Electromagnetic waves}
In  the region where there is no charge and current, maxwell's equations are 
\begin{align}
\div{\vb{E}}&=0\\
\div{\vb{B}}&=0\\
\curl{\vb{E}}&=-\pdv{\vb{B}}{t}\\
\curl{\vb{B}}&=\mu_0\epsilon_0\pdv{\vb{E}}{t}
\end{align}
Applying curl to $(3)$ and $(4)$ 
\[\curl{\qty(\curl{\vb{E}})}=\grad{\qty(\div{\vb{E}})}-\laplacian{\vb{E}}=\curl{\qty(-\pdv{\vb{B}}{t})}=-\pdv{t}\qty(\curl{\vb{B}})=-\mu_)\epsilon_0\pdv[2]{\vb{E}}{t}\]

\[\curl{\qty(\curl{\vb{B}})}=\grad{\qty(\div{\vb{B}})}-\laplacian{\vb{B}}=\curl{\qty(\mu_0\epsilon_0\pdv{\vb{E}}{t})}=\mu_0\epsilon_0\pdv{t}\qty(\curl{\vb{E}})=-\mu_0\epsilon_0\pdv[2]{\vb{B}}{t}\]
Using $(1)$ and $(2)$ \\
\[\boxed{\laplacian{\vb{E}}=\mu_0\epsilon_0\pdv[2]{\vb{E}}{t}\quad\laplacian{\vb{B}}=\mu_0\epsilon_0\pdv[2]{\vb{B}}{t}}\]
If the propagation vector is $\vb{k}$, direction of propagation is in the same direction.
The electric and magnetic fields are given by
\[\boxed{\tilde{\vb{E}}(\vb{r},t)=\tilde{E}_0e^{i(\vb{k}\cdot\vb{r}-\omega t)}\vu{n}}\]
\[\boxed{\tilde{\vb{B}}(\vb{r},t)=\frac{1}{c}\tilde{E}_0e^{i(\vb{k}\cdot\vb{r}-\omega t)}(\vu{k}\cp\vu{n})=\frac{1}{c}\vu{k}\cp\tilde{\vb{E}}}\]
Let the $xy$ plane be the boundary between two linear media. A plane wave of frequency $\omega$ is travelling in the $z$ direction and polarized in the $x$ direction.\\
Incident waves are \[\tilde{\vb{E}}_I(z,t)=\tilde{E}_{0I}e^{i(k_1z-\omega t)}\vu{x}\]
\[\tilde{\vb{B}}_I(z,t)=\frac{1}{v_1}\tilde{E}_{0I}e^{i(k_1z-\omega t)}\vu{y}\]
The reflected waves are  \[\tilde{\vb{E}}_R(z,t)=\tilde{E}_{0R}e^{i(-k_1z-\omega t)}\vu{x}\]
\[\tilde{\vb{B}}_R(z,t)=-\frac{1}{v_1}\tilde{E}_{0I}e^{i(-k_1z-\omega t)}\vu{y}\]
The transmitted waves are  \[\tilde{\vb{E}}_T(z,t)=\tilde{E}_{0T}e^{i(k_2z-\omega t)}\vu{x}\]
\[\tilde{\vb{B}}_T(z,t)=\frac{1}{v_2}\tilde{E}_{0T}e^{i(k_2z-\omega t)}\vu{y}\]
Using boundary conditions,\\
\[\tilde{E}_{0I}+\tilde{E}_{0R}=\tilde{E}_{0T}\]
\[\frac{1}{\mu_1}\qty(\frac{1}{v_1}\tilde{E}_{0I}-\frac{1}{v_1}\tilde{E}_{0R})=\frac{1}{\mu_2}\qty(\frac{1}{v_2}\tilde{E}_{0T})\]
\[\tilde{E}_{0I}-\tilde{E}_{0R}=\beta\tilde{E}_{0T}\qq{ where }\beta=\frac{\mu_1v_1}{\mu_2v_2}\]
\[\tilde{E}_{0R}=\qty(\frac{1-\beta}{1+\beta})\tilde{E}_{0I}\quad\tilde{E}_{0T}=\qty(\frac{2}{1+\beta})\tilde{E}_{0I}\]
\section*{Oblique incidence}
\[\tilde{\vb{E}}_I(\vb{r},t)=\tilde{\vb{E}}_{0I}e^{i(\vb{k}_I\cdot\vb{r}-\omega t)}\quad \tilde{\vb{B}}_I(\vb{r},t)=\frac{1}{v_1}\qty(\vu{k}_I\cp\tilde{\vb{E}}_I)\]
\[\tilde{\vb{E}}_R(\vb{r},t)=\tilde{\vb{E}}_{0R}e^{i(\vb{k}_R\cdot\vb{r}-\omega t)}\quad \tilde{\vb{B}}_R(\vb{r},t)=\frac{1}{v_1}\qty(\vu{k}_R\cp\tilde{\vb{E}}_R)\]
\[\tilde{\vb{E}}_T(\vb{r},t)=\tilde{\vb{E}}_{0T}e^{i(\vb{k}_T\cdot\vb{r}-\omega t)}\quad \tilde{\vb{B}}_T(\vb{r},t)=\frac{1}{v_2}\qty(\vu{k}_T\cp\tilde{\vb{E}}_T)\]
\[k_Iv_1=k_Rv_1=k_Tv_2=\omega\]
Using boundary conditions and matching phases,
\[\vb{k}_I\vdot\vb{r}=\vb{k}_R\vdot\vb{r}=\vb{k}_T\vdot\vb{r}\qq{where} z=0\]
\[x(k_I)_x+y(k_I)_y=x(k_R)_x+y(k_R)_y=x(k_T)_x+y(k_T)_y\]
\[(k_I)_x=(k_R)_x=(k_T)_x\]
\[(k_I)_y=(k_R)_y=(k_T)_y\]
\[k_I\sin\theta_I=k_R\sin\theta_R=k_T\sin\theta_T\]
\[\theta_I=\theta_R\]
\[\frac{\sin\theta_T}{\sin\theta_I}=\frac{v_2}{v_1}=\frac{n_1}{n_2}\]
Boundary conditions are 
\begin{align*}
\epsilon_1\qty(\tilde{\vb{E}}_{0I}+\tilde{\vb{E}}_{0R})_z&=\epsilon_2\qty(\tilde{\vb{E}}_{0T})_z\\
\qty(\tilde{\vb{B}}_{0I}+\tilde{\vb{B}}_{0R})_z&=\qty(\tilde{\vb{B}}_{0T})_z\\
\qty(\tilde{\vb{E}}_{0I}+\tilde{\vb{E}}_{0R})_{x,y}&=\qty(\tilde{\vb{E}}_{0T})_{x,y}\\
\frac{1}{\mu_1}\qty(\tilde{\vb{B}}_{0I}+\tilde{\vb{B}}_{0R})_{x,y}&=\frac{1}{\mu_2}\qty(\tilde{\vb{B}}_{0T})_{x,y}
\end{align*}
If the polarization of incident wave is parallelto the plane of incidence,
\begin{align*}
\epsilon_1(-\tilde{E}_{0I}\sin\theta_I+\tilde{E}_{0R}\sin\theta_R)&=\epsilon_2(-\tilde{E}_{0T}\sin\theta_T)\\
\tilde{E}_{0I}\cos\theta_I+\tilde{E}_{0R}\cos\theta_R&=
\tilde{E}_{0T}\cos\theta_T\\
\frac{1}{\mu_1v_1}(\tilde{E}_{0I}-\tilde{E}_{0R})&=\frac{1}{\mu_2v_2}\tilde{E}_{0T}
\end{align*}
\[(\tilde{E}_{0I}-\tilde{E}_{0R})=\frac{\mu_1v_1}{\mu_2v_2}\tilde{E}_{0T}=\beta\tilde{E}_{0T}\]
\[(\tilde{E}_{0I}+\tilde{E}_{0R})=\alpha\tilde{E}_{0T}\qq{where}\alpha=\frac{\cos\theta_T}{\cos\theta_I}\]
\[\boxed{\tilde{E}_{0R}=\qty(\frac{\alpha-\beta}{\alpha+\beta})\tilde{E}_{0I}\quad\tilde{E}_{0T}=\qty(\frac{2}{\alpha+\beta})\tilde{E}_{0I}}\]

\section{Potentials and fields}

\section{Radiation}

\section{Relativity}
postulates
Simultaneous events.
\[\qq{time dilation} \Delta{\bar{t}}=\frac{\Delta{t}}{\gamma}=\Delta{t}\sqrt{1-\frac{v^2}{c^2}}\]
Lorenz contraction
\[\qq{length contraction} \Delta{\bar{x}}=\gamma\Delta{x}=\frac{\Delta{x}}{\sqrt{1-\frac{v^2}{c^2}}}\]
Lorenz transformations 
\begin{align*}
x_2&=\gamma(x_1-vt_1)\\y_2&=y_1\\z_2&=z_1\\t_2&=\gamma\qty(t_1-\frac{\gamma}{c^2}x_1)
\end{align*}
\begin{align*}
x_1&=\gamma(x_2+vt_2)\\y_1&=y_2\\z_1&=z_2\\t_1&=\gamma\qty(t_2+\frac{\gamma}{c^2}x_2)
\end{align*}
\[v_{AB}=\frac{v_{AC}+v_{CB}}{1+\frac{v_{AC}v_{CB}}{c^2}}\]
\[
\begin{pmatrix}
\bar{x}^0\\ \bar{x}^1\\ \bar{x}^2\\ \bar{x}^3
\end{pmatrix}
=
\begin{pmatrix}
\gamma& -\gamma\beta& 0& 0\\
-\gamma\beta& -\gamma& 0& 0\\
0& 0& 1& 0\\
0& 0& 0& 1
\end{pmatrix}
\begin{pmatrix}
\bar{x}^0\\ \bar{x}^1\\ \bar{x}^2\\ \bar{x}^3
\end{pmatrix}
\]

covariant contravariant
\[a_{\mu}b^{\mu}=a^{\mu}b_{\mu}=-a^{0}b^{0}+a^{1}b^{1}+a^{2}b^{2}+a^{3}b^{3}\]
invariant interval\[\Delta{x}_{\mu}\Delta{x}^{\mu}=\Delta{x}^{\mu}\Delta{x}_{\mu}=-(\Delta{x}^{0})^2+(\Delta{x}^{1})^2+(\Delta{x}^{2})^2+(\Delta{x}^{3})^2=d^2-c^2t^2\]
light like, space like,time like
proper velocity,$\dd{\tau}=\sqrt{1-\frac{u^2}{c^2}}\dd{t}$
\end{document}

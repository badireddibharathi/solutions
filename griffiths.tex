% !TEX TS-program = pdflatex
% !TEX encoding = UTF-8 Unicode

\documentclass[12pt]{article}

\usepackage[utf8]{inputenc} 

\usepackage{geometry}
\geometry{a4paper} 
\geometry{margin=0.25in} 
\geometry{portrait}

\usepackage{amsmath}
\usepackage{physics}
\usepackage{tikz} 

\title{Electrodynamics}
\author{vijayabhaskar badireddi}
%\date{} 

\begin{document}
%\maketitle

\section*{Vectors}
\[\dd{T}=\pdv{T}{x}\dd{x}+\pdv{T}{y}\dd{y}+\pdv{T}{z}\dd{z}=\grad{T}\vdot\dd{\vb{l}}\]
The gradient $\grad{T}$ points in the direction of  maximum increase in the function $T$. The magnitude $|\grad{T}|$ gives the slope along the maximal direction.
\subsection*{Cartesian}
\[\dd{\vb{l}}=\dd{x}\vu{\vb{x}}+\dd{y}\vu{\vb{y}}+\dd{z}\vu{\vb{z}}\quad\dd{\tau}=\dd{x}\dd{y}\dd{z}\]
\[\qq{Gradient}\quad\grad{f}=\pdv{f}{x}\vu{\vb{x}}+\pdv{f}{y}\vu{\vb{y}}+\pdv{f}{z}\vu{\vb{z}}\]
\[\qq{Divergence}\quad\div{\vb{V}}=\pdv{V_x}{x}+\pdv{V_y}{y}+\pdv{V_z}{z}\]
\[\qq{Curl}\quad\curl{\vb{V}}=\qty(\pdv{V_z}{y}-\pdv{V_y}{z})\vu{\vb{x}}+\qty(\pdv{V_x}{z}-\pdv{V_z}{x})\vu{\vb{y}}+\qty(\pdv{V_y}{x}-\pdv{V_x}{y})\vu{\vb{z}}\]
\[\qq{Laplacian}\quad\laplacian{f}=\pdv[2]{f}{x}+\pdv[2]{f}{y}+\pdv[2]{f}{z}\]
\subsection*{Spherical}
\begin{align*}
\vu{r}&=\sin\theta\cos\phi\vu{x}+\sin\theta\cos\phi\vu{y}+\cos\theta\vu{z}\\
\vu{\theta}&=\cos\theta\cos\phi\vu{x}+\cos\theta\sin\phi\vu{y}-\sin\theta\vu{z}\\
\vu{\phi}&=-\sin\phi\vu{x}+\cos\phi\vu{y}
\end{align*}
\[\dd{\vb{l}}=\dd{r}\vu{\vb{r}}+r\dd{\theta}\vu{\vb{\theta}}+r\sin\phi\dd{\phi}\vu{\vb{\phi}}\quad\dd{\tau}=r^2\sin\theta\dd{r}\dd{\theta}\dd{\phi}\]
\[\qq{Gradient}\quad\grad{f}=\pdv{f}{r}\vu{\vb{r}}+\frac{1}{r}\pdv{f}{\theta}\vu{\vb{\theta}}+\frac{1}{r\sin\theta}\pdv{f}{\phi}\vu{\vb{\phi}}\]
\[\qq{Divergence}\quad\div{\vb{V}}=\frac{1}{r^2}\pdv{r}\qty(r^2V_r)+\frac{1}{r\sin\theta}\pdv{\theta}\qty(\sin{\theta}V_{\theta})+\frac{1}{r\sin\theta}\pdv{V_{\phi}}{\phi}\]
\[\qq{Curl}\quad\curl{\vb{V}}=\frac{1}{r\sin\theta}\bqty{\pdv{\theta}\qty(\sin{\theta}V_{\phi})-\pdv{V_{\theta}}{\phi}}\vu{\vb{r}}+\frac{1}{r}\bqty{\frac{1}{\sin\theta}\pdv{V_r}{\phi}-\pdv{r}\qty(rV_{\phi})}\vu{\vb{\theta}}+\frac{1}{r}\bqty{\pdv{r}\qty(rV_{\theta})-\pdv{V_r}{\theta}}\vu{\vb{\phi}}\]
\[\qq{Laplacian}\quad\laplacian{f}=\frac{1}{r^2}\pdv{r}\qty(r^2\pdv{f}{r})+\frac{1}{r^2\sin\theta}\pdv{\theta}\qty(\sin{\theta}\pdv{f}{\theta})+\frac{1}{r^2\sin^2{\theta}}\pdv[2]{f}{\phi}\]
\subsection*{Cylindrical}
\begin{align*}
\vu{s}&=\cos\phi\vu{x}+\sin\phi\vu{y}\\
\vu{\phi}&=-\sin\phi\vu{x}+\cos\phi\vu{y}\\
\vu{z}&=\vu{z}
\end{align*}
\[\dd{\vb{l}}=\dd{s}\vu{\vb{s}}+s\dd{\phi}\vu{\vb{\phi}}+\dd{z}\vu{\vb{z}}\quad\dd{\tau}=s\dd{s}\dd{\phi}\dd{z}\]
\[\qq{Gradient}\quad\grad{f}=\pdv{f}{s}\vu{\vb{s}}+\frac{1}{s}\pdv{f}{\phi}\vu{\vb{\phi}}+\pdv{f}{z}\vu{\vb{z}}\]
\[\qq{Divergence}\quad\div{\vb{V}}=\frac{1}{s}\pdv{s}\qty(sV_s)+\frac{1}{s}\pdv{V_{\phi}}{\phi}+\pdv{V_z}{z}\]
\[\qq{Curl}\quad\curl{\vb{V}}=\bqty{\frac{1}{s}\pdv{V_z}{\phi}-\pdv{V_{\phi}}{z}}\vu{\vb{s}}+\bqty{\pdv{V_s}{z}-\pdv{V_z}{s}}\vu{\vb{\phi}}+\frac{1}{s}\bqty{\pdv{s}\qty(sV_{\phi})-\pdv{V_s}{\phi}}\vu{\vb{z}}\]
\[\qq{Laplacian}\quad\laplacian{f}=\frac{1}{s}\pdv{s}\qty(s\pdv{f}{s})+\frac{1}{s^2}\pdv[2]{f}{\phi}+\pdv[2]{f}{z}\]
\subsection*{vector identities}
\[\vb{A}\vdot\qty(\vb{B}\cp\vb{C})=\vb{B}\vdot\qty(\vb{C}\cp\vb{A})=\vb{C}\vdot\qty(\vb{A}\cp\vb{B})\]
\[\vb{A}\cp\qty(\vb{B}\cp\vb{C})=\vb{B}\qty(\vb{A}\vdot\vb{C})-\vb{C}\qty(\vb{A}\vdot\vb{B})\]
\[\grad{(fg)}=f(\grad{g})+g(\grad{f})\]
\[\grad{(\vb{A}\vdot\vb{B})}=\vb{A}\cp\qty(\curl{\vb{B}})+\vb{B}\cp\qty(\curl{\vb{A}})+\qty(\vb{A}\vdot\nabla)\vb{B}+\qty(\vb{B}\vdot\nabla)\vb{A}\]
\[\div{\qty(f\vb{A})}=f(\div{\vb{A}})+\vb{A}\vdot\qty(\grad{f})\]
\[\div{\qty(\vb{A}\cp\vb{B})}=\vb{B}\vdot\qty(\curl{\vb{A}})-\vb{A}\vdot\qty(\curl{\vb{B}})\]
\[\curl{\qty(f\vb{A})}=f(\curl{\vb{A}})-\vb{A}\cp\qty(\grad{f})\]
\[\curl{(\vb{A}\cp\vb{B})}=\qty(\vb{B}\vdot\nabla)\vb{A}-\qty(\vb{A}\vdot\nabla)\vb{B}+\vb{A}\qty(\div{\vb{B}})-\vb{B}\qty(\div{\vb{A}})\]
\[\div{\curl{\vb{A}}}=0\quad\curl{\grad{f}}=0\quad\curl{\curl{\vb{A}}}=\grad\qty{\curl{\vb{A}}}-\laplacian{\vb{A}}\]
\[\qq{gradient theorem}\quad \int_{\vb{a}}^{\vb{b}}\qty(\grad{f})\vdot\dd{\vb{l}}=f(\vb{b})-f(\vb{a})\]
\[\qq{divergence theorem} \int\qty(\div{\vb{A}})\dd{\tau}=\oint\vb{A}\vdot\dd{\vb{a}}\]
\[\qq{curl theorem} \int\qty(\curl{\vb{A}})\vdot\dd{\vb{a}}=\oint\vb{A}\vdot\dd{\vb{l}}\]
\section{Vector Analysis}
\subsection*{Dirac delta function}
\[\div{\qty(\frac{\vb{r}}{r^2})}=4\pi\delta^3(\vb{r})\]
\[\laplacian{\qty(\frac{1}{r})}=-4\pi\delta^3(\vb{r})\]
\section{Electrostatics}
\[\qq{Coulomb's law}\vb{F}_{12}=\frac{1}{4\pi\epsilon_0}\frac{q_1q_2(\vb{r}_1-\vb{r}_2)}{\vqty{\vb{r}_1-\vb{r}_2}^3}\]
\[\qq{Gauss law}\quad \oint\vb{E}\vdot\dd{\vb{a}}=\frac{Q_{enc}}{\epsilon_0}\]
\[\div{\vb{E}}=\frac{\rho}{\epsilon_0}\]
\[\div{\qty(\frac{\vb{r}-\vb{r}'}{\vqty{\vb{r}-\vb{r}'}^2})}=4\pi\delta^3(\vb{r}-\vb{r}')\]
\[\curl{\vb{E}}=0\quad \oint\vb{E}\vdot\dd{\vb{l}}=0\]
\[V(\vb{r})=-\int_{\infty}^{\vb{r}}\vb{E}\vdot\dd{\vb{l}}\quad\vb{E}=-\grad{V}\]
\[\qq{Poisson's equation}\quad\laplacian{V}=-\frac{\rho}{\epsilon_0}\]
\[\qq{Laplace's equation} \quad\laplacian{V}=0\]
\[\qq{boundary conditions} \vb{E}^{\parallel}_{above}=\vb{E}^{\parallel}_{below}\quad \vb{E}_{above}-\vb{E}_{below}=\frac{\sigma}{\epsilon_0}\vu{n} \]
\[W=\frac{1}{2}\int\rho{V}\dd{\tau}=\frac{\epsilon_0}{2}\int{E^2}\dd\tau\]
\section{Special techniques}
\[\qq{First uniqueness theorem}\]
The solution to Laplace's equation in some volume is uniquely determined by specifying potential on the boundary surface.
\[\qq{Second uniqueness theorem}\]
In a volume surrounded by conductors and containing a specified charge density, the electric field is uniquely determined if the total charge on each conductor is given.
\subsection*{method of images}
\subsection*{separation of variables}
\subsection*{multipole expansion}
\section{Electric fields in matter}
\[\vb{N}=\vb{p}\cp\vb{E}\quad\vb{F}=\qty(\vb{p}\vdot\nabla)\vb{E}\quad U=-\vb{p}\vdot\vb{E}\]
\subsection*{polarization}
\[\sigma_b=\vb{P}\vdot\vu{n}\quad\rho_b=-\div{\vb{P}}\]
\subsection*{displacement}
\subsection*{misc}
\subsection*{Coulomb's law}
\[\vb{E}=\frac{1}{4\pi\epsilon_0}\frac{q_1q_2(\vb{r}_1-\vb{r}_2)}{|\vb{r}_1-\vb{r}_2|^3}\]
\[V(\vb{r})=-\int_\infty^{\vb{r}}\vb{E}\cdot\dd\vb{r}\]
\[\vb{E}=-\grad{V}\]
\subsection*{Gauss' law}
\[\oint_S{\vb{E}\cdot\dd{\vb{a}}}=\frac{Q_{enc}}{\epsilon_0}\]
\[\laplacian{V}=-\frac{\rho}{\epsilon_0}\]
\[\pdv[2]{V}{x}+\pdv[2]{V}{y}+\pdv[2]{V}{z}=0\]
\[\frac{1}{r^2}\pdv{r}\qty(r\pdv{V}{r})+\frac{1}{r\sin\theta}\pdv{\theta}\qty(\sin\theta\pdv{V}{\theta})+\frac{1}{r^2\sin^2\theta}\pdv[2]{V}{\phi}=0\]

\[\int_{v}(\div{\vb{A}})\dd\tau=\oint_{s}\vb{A}\cdot\dd{\vb{a}}\]
\[\int_{v}(\div{\vb{A}})\dd\tau=\oint_{s}\vb{A}\cdot\dd{\vb{a}}\]
\[\grad{(\vb{a}\cdot\vb{b})}=\vb{a}\times(\curl{\vb{b}})+\vb{b}\times(\curl{\vb{a}})+(\vb{a}\cdot\nabla)\vb{b}+(\vb{b}\cdot\nabla)\vb{a}\]
\[\div{(f\vb{a})}=\vb{a}\cdot\grad{f}+f\div{\vb{a}}\]
\[\div{(\vb{a}\times\vb{b})}=\vb{b}\cdot(\curl{\vb{a}})-\vb{a}\cdot(\curl{\vb{b}})\]
\section{Magnetostatics}
\[\qq{Lorenz force law}\vb{F}=q(\vb{E}+\vb{v}\cp\vb{B})\]
\[\qq{Lorenz force law}\vb{F}=\int I(\dd\vb{l}\cp\vb{B})\]
\[\vb{F}=\int (\vb{K}\cp\vb{B})\dd{a}=\int (\sigma\vb{v}\cp\vb{B})\dd{a}\]
\[\vb{F}=\int (\vb{J}\cp\vb{B})\dd{v}=\int (\rho\vb{v}\cp\vb{B})\dd{v}\]
\[\vb{B}(\vb{r})=\frac{\mu_0}{4\pi}\int\frac{\vb{I}\cp(\vb{r}-\vb{r}')}{|\vb{r}-\vb{r}'|^2}\dd{l'}=\frac{\mu_0I}{4\pi}\int\frac{\dd\vb{l'}\cp(\vb{r}-\vb{r}')}{|\vb{r}-\vb{r}'|^2}\]
\[\curl{\vb{B}}=\mu_0\vb{J}\]
\[\div{\vb{B}}=0\]
\section{Conservation Laws}
\[\qq{Equation of continuity} \pdv{\rho}{t}=-\div{\vb{J}}\]
\[\dv{W}{t}=\int_{v}(\vb{E}\vdot{\vb{J}})\dd{\tau}=-\dv{t}\int_{v}\frac{1}{2}\qty(\epsilon_0E^2+\frac{1}{\mu_0}B^2)\dd{\tau}-\frac{1}{\mu_0}\oint(\vb{E}\cp\vb{B})\vdot\dd{\vb{a}}\]
\[\qq{Poynting vector}\vb{S}=\frac{1}{\mu_0}(\vb{E}\cp\vb{B})\]
\[\qq{Maxwell's stress tensor} T_{ij}=\epsilon_0\qty(E_iE_j-\frac{1}{2}\delta_{ij}E^2)+\frac{1}{\mu_0}\qty(B_iB_j-\frac{1}{2}\delta_{ij}B^2)\]
\[\qq{force per unit volume} \vb{f}=\div{\vb{T}}-\epsilon_0\mu_0\pdv{\vb{S}}{t}\]
\[\vb{F}=\oint_s\vb{T}\vdot\dd{\vb{a}}-\epsilon_0\mu_0\dv{t}\int_v\vb{S}\dd{\tau}\]
\[\vb{p}_{em}=\mu_0\epsilon_0\int_v\vb{S}\dd{\tau}\]
\[\mathcal{P}=\mu_0\epsilon_0\vb{S}\]
\[\pdv{t}(\mathcal{P}_{mech}+\mathcal{P}_{em})=\div{\vb{T}}\]
\section{Electrodynamics}
\[\qq{Ohm's law} \vb{J}=\sigma\vb{E}\]
\[\mathcal{E}=-\dv{\Phi}{t}\]
\[\curl{\vb{E}}=-\pdv{\vb{B}}{t}\]
\[\Phi_2=\int\vb{B}_1\vdot\dd{\vb{a}_2}\]
\[M_{21}=M_{21}=\frac{\mu_0I_1I_2}{4\pi}\oint\oint\frac{\dd{\vb{l}_1}\vdot\dd{\vb{l}_2}}{|\vb{r}-\vb{r}' |}\]
\[W=\frac{1}{2}LI^2=\frac{1}{2}\int B^2\dd{\tau}\]
\subsection{Boundary conditions}
\[\qq{Gauss law for displacement}\oint_{S}\vb{D}\vdot\dd{\vb{a}}=Q_{free,enc}\]
\[\qq{no magnetic monopoles}\oint_{S}\vb{B}\vdot\dd{\vb{a}}=0\]
\[\oint\vb{E}\vdot\dd{\vb{l}}=-\dv{t}\int\vb{B}\vdot\dd{\vb{a}}\]
\[\oint\vb{H}\vdot\dd{\vb{l}}=I_{free,enc}+\dv{t}\int\vb{D}\vdot\dd{\vb{a}}\]
\section{Relativity}
\[\qq{time dilation} \Delta{\bar{t}}=\frac{\Delta{t}}{\gamma}=\Delta{t}\sqrt{1-\frac{v^2}{c^2}}\]

\end{document}

\section*{foundations}
The state of a quantum mechanical system by an element of an abstract vector space called the state space. These elements are called kets.\\
An observable is a Hermitian operator for which there is an orthonormal basis of the state space that consists of eigen vectors of this operator. 
The dual space consists of all linear functionals acting on the state space. Elements of this dual space are called bras.
There is a scalar product, which has the following properties.
\begin{align*}
\braket{\phi}{\psi}=\braket{\psi}{\phi}^*\\
\braket{\psi}{\lambda_1\phi_1+\lambda_2\phi_2}=\lambda_1\braket{\psi}{\phi_1}+\lambda_2\braket{\psi}{\phi_2}\\
\braket{\lambda_1\phi_1+\lambda_2\phi_2}{\psi}=\lambda_1^*\braket{\phi_1}{\psi}+\lambda_2^*\braket{\phi_2}{\psi}\\
\braket{\psi}{\psi}=|\braket{\psi}{\psi}|=\geq 0
\end{align*}
\subsection*{Postulates of quantum mechanics}
The state of a physical system at time $t=t_0$ is defined by specifying a ket $\ket{\psi(t_0)}$ belonging to the state space.
A measurable physical quantity $A$ is described by an observable $\hat{A}$ acting on kets of a state space.Possible results of measurement of a physical quantity $A$ are the eigen values of the corresponding observable $\hat{A}$.
Let $A$ be a physical quantity with a corresponding observable $\hat{A}$.Suppose the system is in a normalized state $\ket{\psi}$. When $A$ is measured, the probability $P(a_n)$ of obtaining eigen value $a_n$ of $\hat{A}$ is \[P(a_n)=\sum_{i=1}^{g_n}|\braket{u^i_n}{\psi}|^2\] where $g_n$ is the degeneracy of $a_n$ and $\ket{u_n^1},\ket{u_n^2},\ket{u_n^3}\ldots$ form an orthonormal basis of the subspace spanned by eigen vectors of $\hat{A}$ with eigen value $a_n$.
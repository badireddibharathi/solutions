% !TeX root = main.tex
\section*{emfield}
Electric field $\vb{E}$ and magnetic field $\vb{B}$
are given by 
\begin{align*}
\vb{E}=-\grad{\phi}-\pdv{\vb{A}}{t}\qq{and}\vb{B}=\curl{\vb{A}}
\end{align*}
When $\vb{E}$ and $\vb{B}$ are given, $\phi$ and $\vb{A}$ are not unique.
\begin{align*}
\phi'=\phi-\pdv{f(\vb{r},t)}{t}\qq{and}\vb{A'}=\vb{A}+\grad{f(\vb{r},t)}
\end{align*}
\[\qq{Symmetric gauge} \vb{A}=-\frac{1}{2}\vb{r}\cp\vb{B}\]
\[\hat{H}=\frac{1}{2m}(\hat{p}-q\mathbf{A})\cdot(\hat{p}-q\mathbf{A})+ q\phi\]
\[\qq{probability density}\rho=\Psi^*(\vb{r},t)\Psi(\vb{r},t)\]
\[\qq{probability current density}\vb{s}=\frac{1}{2m}\qty[\frac{\hbar}{i}(\Psi^*(\vb{r},t)\grad{\Psi(\vb{r},t)}-\Psi(\vb{r},t)\grad{\Psi^*(\vb{r},t)})-2q\vb{A}\Psi^*(\vb{r},t)\Psi(\vb{r},t)]\]
\[\vb{s}=\frac{1}{2m}\qty[\frac{\hbar}{i}(\Psi^*(\vb{r},t)\grad{\Psi(\vb{r},t)}-\Psi(\vb{r},t)\grad{\Psi^*(\vb{r},t)})-2q\vb{A}\Psi^*(\vb{r},t)\Psi(\vb{r},t)]+\frac{\mu}{S}\curl{(\Psi^*(\vb{r},t)\hat{S}\Psi(\vb{r},t))}\]
\subsection*{equation of motion}
Use \[\vb{E}=-\grad{\phi}-\pdv{\vb{A}}{t}\qq{and}\vb{B}=\curl{\vb{A}}\]
and Hamiltonian \[\hat{H}=\frac{1}{2m}(\hat{p}-q\mathbf{A})\cdot(\hat{p}-q\mathbf{A})+ q\phi\]
and the Hamilton's equations of motion \[\dot{\vb{r}}=\pdv{H}{\vb{p}}\quad\dot{\vb{p}}=-\pdv{H}{\vb{r}}\]
\subsection*{9.1}
The classical equation of motion is \[m\vb{a}=q\vb{E}+q(\vb{v}\cp\vb{B})\]
Given scalar potential $\phi(\vb{r},t)$ and vector potential $\vb{A}(\vb{r},t)$, 
\[\dot{\vb{r}}=\pdv{H}{\vb{p}}=\frac{1}{m}(\vb{p}-q\vb{A})=\vb{v}\]
\[\ddot{\vb{r}}=\dv{t}\qty(\frac{1}{m}(\vb{p}-q\vb{A}))=\frac{1}{m}(\dot{\vb{p}}-q\dot{\vb{A}})\]
\[\dot{\vb{p}}=-\pdv{H}{\vb{r}}=-\grad{H}=-\grad{\qty[\frac{1}{2m}(\hat{p}-q\mathbf{A})\cdot(\hat{p}-q\mathbf{A})+ q\phi]}\]
%\[\dot{\vb{p}}=-\grad{\qty[\frac{1}{2m}(\hat{p}-q\mathbf{A})\cdot(\hat{p}-q\mathbf{A})]}+ q\grad{\phi}}\]
\subsection*{9.2}
\newpage
\subsection*{9.3}
\newpage
\subsection*{9.16 Aharonov-Bohm effect}
An electron is confined to move on a ring of radius $R$. At the center of ring there is a circular region with uniform and constant magnetic field perpendicular to the plane of ring. Total magnetic flux through the region is $\Phi$. 
\begin{align*}
\Phi=\int\vb{B}\cdot\dd\vb{S}=\int\curl\vb{A}\cdot\dd\vb{S}
\\
\qq{using Stoke's law}
\Phi=\int\curl\vb{A}\cdot\dd\vb{S}=\oint\vb{A}\cdot\dd\vb{l}
\\
\qq{By symmetry,}A_{\phi}\ne 0\qq{and}A_{\rho}=0\quad A_z=0
\\
\Phi=\oint\vb{A}\cdot\dd\vb{l}=\int_0^{2\pi}A_{\phi}R\dd\phi=A_{\phi}2\pi R
\qq{Hence} A_{\phi}=\frac{\Phi}{2\pi R}
\\
\qq{In cylindrical coordinates}
\nabla=\vu{\bm\rho}\pdv{\rho}+\vu{\bm\phi}\frac{1}{\rho}\pdv{\phi}+\vu{z}\pdv{z}\\
\qq{In this problem electron is confined to move on a ring.}z=\text{constant}\qq{and}\rho=R
\end{align*}
\begin{align*}
\hat{H}=\frac{1}{2m}\qty[(\hat{p}-e\vb{A})^2]=\frac{1}{2m}\qty[(-i\hbar\nabla-e\vb{A})^2]=\frac{1}{2m}\qty(-i\hbar\frac{1}{\rho}\pdv{\phi}-e\vb{A})^2
\end{align*}



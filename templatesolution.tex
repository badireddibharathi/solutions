% !TEX TS-program = pdflatex
% !TEX encoding = UTF-8 Unicode

\documentclass[12pt] {article}
\usepackage{physics}
\usepackage{amsmath}
\usepackage{geometry}
\geometry{a4paper}
\geometry{margin=0.25in}
\geometry{portrait}


\title{templatesolution}
\author{vijayabhaskar badireddi}

\begin{document}
        
\section*{solution}
A charge $q_1$ is located at point $\vb{r}_1$.
Electric field due to this charge $q_1$ at point $\vb{r}$ is
\[\vb{E}=\frac{1}{4\pi\epsilon_0}\frac{(\vb{r}-\vb{r}_1)}{|\vb{r}-\vb{r}_1|^3}\]
When a large number of charges are collected on a surface, it is called a surface charge.
A particle with initial velocity $\vb{v}_i$ is subjected to an acceleration $\vb{a}$.
After a time $t$, velocity of the particle will be $\vb{v}_f=\vb{v}_i+\vb{a}t$.
During this time interval, displacement of particle will be $\Delta{\vb{r}}=\vb{v}_it+\frac{1}{2}\vb{a}t^2$
A particle is projected with a velocity $\vb{v}_i$, at angle $\theta$ with
 horizontal in a uniform gravitational field $\vb{a}=-g\vu{z}$. The particle returns to 
 horizontal level after a time $t=\frac{2v_i\sin\theta}{g}$ at a horizontal distance
  $R=\frac{v_i^2\sin2\theta}{g}$ from the point of projection.
\subsection{mechanics}
The expectation value of an operator $\hat{A}$ is $\expval{\hat{A}}{\psi}$. Outer product 
of two kets $\ket{\psi}$ and $\ket{\phi}$ is $\op{\psi}{\phi}$.
Normalizing a ket is nothing but making it a unit vector.
\[\div{\vb{E}}=\frac{1}{\epsilon_0}\rho\]
\[\curl{\vb{E}}=-\pdv{\vb{B}}{t}\]
\end{document}
% !TEX TS-program = pdflatex
% !TEX encoding = UTF-8 Unicode

\documentclass[12pt]{article}

\usepackage[utf8]{inputenc} 

\usepackage{geometry}
\geometry{a4paper} 
\geometry{margin=0.25in} 
\geometry{portrait}

\usepackage{amsmath}
\usepackage{physics}
\usepackage{tikz} 

\title{quantum mechanics}
\author{vijayabhaskar badireddi}
%\date{} 

\begin{document}
%\maketitle
\section{Mathematics}
\section{Schrodinger Equation}
Probability that a particle is in a volume $\dd[3]{r}\dd{x}\dd{y}\dd{z}$ is $\dd{P}=\Psi^*(r,t)\Psi(r,t)\dd[3]{r}$.
\[\int\dd{P}=\int\Psi^*(r,t)\Psi(r,t)\dd[3]{r}=1\] 
The wave function msut be square-integrable, that is \[\int\Psi^*(r,t)\Psi(r,t)\dd[3]{r}<0\]
The time-dependent Schroedinger equation \[i\hbar\pdv{\Psi(\vb{r},t)}{t}=-\frac{\hbar^2}{2m}\laplacian{\Psi(\vb{r},t)}+V(\vb{r},t)\Psi(\vb{r},t)\]
Linearity of Schroedinger equation.
The time-independent Schroedinger equation \[i\hbar\pdv{\psi(\vb{r})}{t}=-\frac{\hbar^2}{2m}\laplacian{\psi(\vb{r})}+V(\vb{r})\psi(\vb{r})\]
\[\Psi(\vb{r},t)=\psi(r)e^{-iEt/\hbar}\]
wave packet or a superposition of plane waves.
Inner product or scalar product\[(\phi,\psi)=\int\phi^*(\vb{r})\psi(\vb{r})\dd[3]{r}\]
\[\expval{\hat{A}}=\int\phi^*(\vb{r})\hat{A}\psi(\vb{r})\dd[3]{r}\]
\[\Delta{\hat{A}}=\sqrt{\expval{\hat{A}^2}-\expval{\hat{A}}^2}\]
\[\qq{Hamiltonian operator}i\hbar\pdv{\Psi(\vb{r},t)}{t}=\hat{H}\Psi(\vb{r},t)=-\frac{\hbar^2}{2m}\laplacian{\Psi(\vb{r},t)}+V(\vb{r},t)\Psi(\vb{r},t)\]
\[\qq{equation of continuity}\pdv{\rho(\vb{r},t)}{t}+\div{\vb{J}(\vb{r},t)}=0\]
\[\qq{probability current density}\vb{J}(\vb{r},t)=\frac{\hbar}{2mi}\qty[\Psi^*(\grad\Psi)-\Psi(\grad\Psi^*)]\]
\subsection*{Particle in a one-dimensional infinite well}
Potential is defined as \[V(x)=0\qq{if} 0<x<a V(x)=\infty \qq{if}x>a \qq{or if }x<0\]
Where $V(x)=\infty$ the wave function $\psi(x) =0$. \\Applying Schroedinger's equation in the region of well \[-\frac{\hbar^2}{2m}\dv[2]{\psi}{x}+V(x)\psi=E\psi\]
\[-\frac{\hbar^2}{2m}\dv[2]{\psi}{x}=E\psi\qq{since} V(x)=0\]
\[\qq{Rewrite as}\dv[2]{\psi}{x}+\frac{2mE}{\hbar^2}\psi=0\]
\[\dv[2]{\psi}{x}+k^2\psi=0\qq{where} k^2=\frac{2mE}{\hbar^2}\]
\[\qq{General solution is}\psi(x)=A\sin{kx}+B\cos{kx}\]
Using the boundary condition $\psi(x)=0$ if $x=0$, we obtain $B=0$
 or $\psi(x)=A\sin{kx}$ \\
Using the boundary condition $\psi(x)=0$ if $x=a$, we obtain $\sin{ka}=0$ or $ka=n\pi$ where $n=1,2,3\ldots$\\
Hence the General solution can be written as $\displaystyle\psi(x)=A\sin{\left(\frac{n\pi x}{a}\right)}$.\\
Using the normalization condition $\displaystyle\int_{-\infty}^{+\infty}\psi^*(x)\psi(x)\dd{x}=1$,\\
\[\displaystyle\int_{0}^{a}A^2\sin^2{\left(\frac{n\pi x}{a}\right)}\dd{x}=1\]
\[\displaystyle A^2\int_{0}^{a}\frac{1-\cos{\left(\frac{2n\pi x}{a}\right)}}{2}\dd{x}=1\]
\[\displaystyle \frac{A^2}{2}\qty[\int_{0}^{a}1\dd{x}-\int_{0}^{a}\cos{\left(\frac{2n\pi x}{a}\right)}\dd{x}]=1\]
\[\displaystyle \frac{A^2}{2}\qty[a-\frac{a}{2n\pi}\sin{\left(\frac{2n\pi x}{a}\right)}\eval_{0}^{a}]=1\]
Hence $\displaystyle A^2=\frac{2}{a}$ or $A=\sqrt{\frac{2}{a}}$\\
\[\qq{Hence general solution is}\psi(x)=\sqrt{\frac{2}{a}}\sin{\left(	\frac{n\pi x}{a}\right)}\]
\[k^2=\frac{n^2\pi^2}{a^2}=\frac{2mE_n}{\hbar^2}\]
Energy levels are given by \[E_n=\frac{n^2\hbar^2\pi^2}{2ma^2}\]
\subsection*{symmetric well}
Given potential is \(V(x)=0\) if    \(-a<x<+a\)
and\(V(x)=\infty\) if \(+a<x\) or \(x<-a\)\\
Applying the Schroedinger equation, $\hat{H}\psi=E\psi$ where
$\hat{H}=\frac{\hat{p}^2}{2m}+V(x)$.\\ substituting $\hat{p}=-i\hbar\frac{d}{dx}$
\[-\frac{\hbar^2}{2m}\frac{d^2\psi}{dx^2}=E\psi\]
\[\frac{d^2\psi}{dx^2}=-\frac{2mE}{\hbar^2}\psi=-k^2\psi where k=\sqrt{\frac{2mE}{\hbar^2}}\]
\text{Using the boundary conditions,} $\psi(x)=0$ \text{if} $x=-a$ \text{or if} $x=+a$,\\
Solution of the above equation is \[\psi(x)=A\sin{\left(k(x+a)\right)}=A\sin{\left(\frac{n\pi(x+a)}{2a}\right)}\text{ where } n=1,2,3\ldots\]\\
Normalizing the wave equation, $$A=\sqrt{\frac{1}{a}}.$$
\[\text{ Hence solutions are }  \psi_{n}(x)=\frac{1}{\sqrt{a}}\sin{\left(\frac{n\pi(x+a)}{2a}\right)}\] where$$ n=1,2,3\ldots$$
$$k=\frac{n\pi}{2a}=\sqrt{\frac{2mE_{n}}{\hbar^2}}$$
\[\frac{n^2\pi^2}{4a^2}=\frac{2mE_{n}}{\hbar^2}\qq{hence} E_{n}=\frac{n^2\hbar^2\pi^2}{8ma^2}\]

Each of the above wave functions are energy eigen functions with energy eigen values $$E_{n}=\frac{n^2\hbar^2\pi^2}{8ma^2}.$$
Minimum possible energy occurs for n=1,   $$E_{1}=\frac{\hbar^2\pi^2}{8ma^2} $$
The corresponding eigenfunction is $$\psi_{1}(x)=\frac{1}{\sqrt{a}}\sin{\left(\frac{\pi(x+a)}{2a}\right)}$$
Probability of finding the particle in the region \( x\) to \(x+dx\), is \[P=\int_{x}^{x+dx}\psi_{1}^{*}(x)\psi_{1}(x)dx\]
\[P=\psi_{1}^{*}(x)\psi_{1}(x)dx=\frac{1}{a}\sin^{2}{\left(\frac{\pi(x+a)}{2a}\right)}dx\]
The state of system is \[\psi(x)=\frac{1}{\sqrt{2}}\left(\psi_{1}(x)+\psi_{2}(x) \right )\]
The average of position \(\hat{x}\) over many measurements is 
\[\expval{\hat{x}}=\int_{-a}^{+a}\psi^{*}(x)\hat{x}\psi(x)dx
\langle\hat{x}\rangle=\frac{1}{2}\int_{-a}^{+a}\left(\psi_{1}(x)+\psi_{2}(x) \right )\hat{x}\left(\psi_{1}(x)+\psi_{2}(x) \right )dx\]
\[\expval{\hat{x}}=\frac{1}{2}\left[\int_{-a}^{+a}\psi_{1}^{*}(x)\hat{x}\psi_{1}(x)dx+\int_{-a}^{+a}\psi_{2}^{*}(x)\hat{x}\psi_{1}(x) dx+\int_{-a}^{+a}\psi_{1}^{*}(x)\hat{x}\psi_{2}(x)dx+\int_{-a}^{+a}\psi_{2}^{*}(x) \hat{x}\psi_{2}(x) dx \right ]\]
\subsection{Ehrenfest's theorem}
\[\qq{relation 1}\dv{\expval{\hat{x}}}{t}=\frac{\expval{\hat{p}}}{m}\]
\[\qq{relation 2}\dv{\expval{p}}{t}=\expval{-\dv{V}{x}}\]
\[\qq{Time-independent Schrodinger equation}i\hbar\pdv{\Psi(x,t)}{t}=-\frac{\hbar^2}{2m}\dv[2]{\Psi(x,t)}{x}+V(x)\Psi(x,t)\]
\[\qq{Time-independent Schrodinger equation}-i\hbar\pdv{\Psi^*(x,t)}{t}=-\frac{\hbar^2}{2m}\dv[2]{\Psi^*(x,t)}{x}+V(x)\Psi^*(x,t)\]
What is the meaning of square integrable ?
\[\int_{-\infty}^{+\infty}\Psi^*(x,t)\Psi(x,t)\dd{x}=\text{finite}\]
Hence \[\lim_{x\to\infty}|\Psi(x,t)|^2=\lim_{x\to -\infty}|\Psi(x,t)|^2=0\]
The expectation value of $\hat{x}$ is \[\expval{\hat{x}}=\int_{-\infty}^{+\infty}\Psi^*(x,t)\hat{x}\Psi(x,t)\dd{x}=\int_{-\infty}^{+\infty}\Psi^*(x,t)x\Psi(x,t)\dd{x}\]
Differentiate the above expression with respect to time.
\[\dv{\expval{\hat{x}}}{t}=\int_{-\infty}^{+\infty}\pdv{\Psi^*(x,t)}{t}x\Psi(x,t)\dd{x}+\int_{-\infty}^{+\infty}\Psi^*(x,t)x\pdv{\Psi(x,t)}{t}\dd{x}\]

\section{Foundations}
The state of a quantum mechanical system by an element of an abstract vector space called the state space. These elements are called kets.\\
An observable is a Hermitian operator for which there is an orthonormal basis of the state space that consists of eigen vectors of this operator. 
The dual space consists of all linear functionals acting on the state space. Elements of this dual space are called bras.
There is a scalar product, which has the following properties.
\begin{align*}
\braket{\phi}{\psi}=\braket{\psi}{\phi}^*\\
\braket{\psi}{\lambda_1\phi_1+\lambda_2\phi_2}=\lambda_1\braket{\psi}{\phi_1}+\lambda_2\braket{\psi}{\phi_2}\\
\braket{\lambda_1\phi_1+\lambda_2\phi_2}{\psi}=\lambda_1^*\braket{\phi_1}{\psi}+\lambda_2^*\braket{\phi_2}{\psi}\\
\braket{\psi}{\psi}=|\braket{\psi}{\psi}|=\geq 0
\end{align*}
\subsection*{Postulates of quantum mechanics}
The state of a physical system at time $t=t_0$ is defined by specifying a ket $\ket{\psi(t_0)}$ belonging to the state space.
A measurable physical quantity $A$ is described by an observable $\hat{A}$ acting on kets of a state space.Possible results of measurement of a physical quantity $A$ are the eigen values of the corresponding observable $\hat{A}$.
Let $A$ be a physical quantity with a corresponding observable $\hat{A}$.Suppose the system is in a normalized state $\ket{\psi}$. When $A$ is measured, the probability $P(a_n)$ of obtaining eigen value $a_n$ of $\hat{A}$ is \[P(a_n)=\sum_{i=1}^{g_n}|\braket{u^i_n}{\psi}|^2\] where $g_n$ is the degeneracy of $a_n$ and $\ket{u_n^1},\ket{u_n^2},\ket{u_n^3}\ldots$ form an orthonormal basis of the subspace spanned by eigen vectors of $\hat{A}$ with eigen value $a_n$.
\section{Harmonic oscillator}
\[\hat{H}\psi=E\psi\]
\[-\frac{\hbar^2}{2m}\dv[2]{\psi}{x}+V(x)\psi=E\psi\]
Since the potential is $V(s)=\frac{1}{2}kx^2=\frac{1}{2}m\omega^2x^2$
\[-\frac{\hbar^2}{2m}\dv[2]{\psi}{x}+\frac{1}{2}m\omega^2x^2\psi=E\psi\]
\[\dv[2]{\psi}{x}-\frac{m^2\omega^2}{\hbar^2}x^2\psi=-\frac{2Em}{\hbar^2}\psi\]
Redefine $\zeta=\sqrt{\frac{m\omega}{\hbar}}x$.
\[\dv[2]{\psi}{x}=\frac{m\omega}{\hbar}\dv[2]{\psi}{\zeta}\]
\[\frac{m\omega}{\hbar}\dv[2]{\psi}{\zeta}-\frac{m\omega}{\hbar}\zeta^2\psi=-\frac{2Em}{\hbar^2}\psi\]
\[\dv[2]{\psi}{\zeta}-\zeta^2\psi=-\frac{\hbar}{m\omega}\frac{2Em}{\hbar^2}\psi=-\frac{2E}{\hbar\omega}\psi\]
\[\dv[2]{\psi}{\zeta}+\qty(\frac{2E}{\hbar\omega}-\zeta^2)\psi=0\]
Substitute $\psi=H(\zeta)e^{-\zeta^2/2}$	
\section{Angular Momentum}
Angular momentum operator is defined by $\vb{\vu{L}}=\vb{\vu{r}}\cp\vb{\vu{p}}$.
\[\hat{L}_x=\hat{y}\hat{p}_z-\hat{z}\hat{p}_y\quad\hat{L}_y=\hat{z}\hat{p}_x-\hat{x}\hat{p}_z\quad\hat{L}_z=\hat{x}\hat{p}_y-\hat{y}\hat{p}_x\]
\[\comm{\hat{L}_i}{\hat{L}_j}=i\hbar\epsilon_{ijk}\hat{L}_k\]
\[\comm{\hat{L}^2}{\vb{\vu{L}}}=0\quad\comm{\hat{L}_i}{\hat{p}^2}=0\quad\comm{\hat{L}_i}{\hat{r}^2}=0\quad\comm{\hat{L}_i}{\vb{\vu{r}}\vdot\vb{\vu{p}}}=0\]
\[\comm{\hat{L}_i}{\hat{r}_j}=i\hbar\sum_{i}\epsilon_{ijk}\hat{L}_k\]
\[\comm{\hat{L}_i}{\hat{p}_j}=i\hbar\sum_{i}\epsilon_{ijk}\hat{p}_k\]
\[\qq{The lowering and raising operators are } \hat{L}_{+}=\hat{L}_x+i\hat{L}_y \qq{and} \hat{L}_{-}=\hat{L}_x-i\hat{L}_y\]
\[\hat{L}_x=\frac{\hat{L}_++\hat{L}_-}{2} \qq{and} \hat{L}_y=\frac{\hat{L}_+-\hat{L}_-}{2i}\]
\[\hat{L}_+=\hat{L}_-^{\dag}\]
\[\hat{L}^2=\hat{L}_z^2+\frac{1}{2}\qty(\hat{L}_+\hat{L}_-+\hat{L}_-\hat{L}_+)\]
\[\hat{L}_+\hat{L}_-=\hat{L}^2-\hat{L}_z^2+\hbar\hat{L}_z\]
\[\hat{L}_-\hat{L}_+=\hat{L}^2-\hat{L}_z^2\pm\hbar\hat{L}_z\]
\[\comm{\hat{L}^2}{\hat{L}_{\pm}}=0\]
\[\comm{\hat{L}_z}{\hat{L}_{\pm}}=\pm\hbar\hat{L}_{\pm}\]
\[\comm{\hat{L}_+}{\hat{L}_-}=2\hbar\hat{L}_z\]
\[\hat{L}^2\ket{lm}=l(l+1)\hbar^2\ket{lm}\]
\[\hat{L}_z\ket{lm}=m\hbar\ket{lm}\]
\[\hat{L}_{\pm}\ket{lm}=\sqrt{l(l+1)-m(m\pm1)}\hbar\ket{lm\pm1}=\sqrt{(l\mp m)(l\pm m+1)}\hbar\ket{lm\pm1}\]
\[m=-l,-l+1,\ldots,0,\ldots,l-1,l\]
\[\braket{l_1m_1}{l_2m_2}=\delta_{l_1l_2}\delta_{m_1m_2}\]
\[\sum_{l=0}^{\infty}\sum_{m=-l}^{+l}=1\]
\section{Spin}
\[\hat{S}^{2}\ket{\alpha}=s(s+1)\hbar^2\ket{\alpha}\]
\[\comm{\hat{S}_{x}}{\hat{S}_{y}}=i\hbar\hat{S}_{z}\]
\[\comm{\hat{S}_{y}}{\hat{S}_{z}}=i\hbar\hat{S}_{x}\]
\[\comm{\hat{S}_{z}}{\hat{S}_{x}}=i\hbar\hat{S}_{y}\]
\[\comm{\hat{S}_{i}}{\hat{S}_{j}}=i\hbar\epsilon_{ijk}\hat{S}_{k}\]
\[\vb{S}=\frac{\hbar}{2}\vb{\sigma}\]
\[\hat{S}_{z}\ket{\alpha}=m_{s}\hbar\ket{\alpha}\]
$\ket{+}$ and $\ket{-}$ are the eigen vectors of $\hat{{S}}_z$ operator.
\[\hat{{S}}_z\ket{+}=\frac{\hbar}{2}\ket{+}\quad\hat{{S}}_z\ket{-}=-\frac{\hbar}{2}\ket{-}\]
\section{Hydrogen atom}
When a particle of mass $m$ is placed in a central potential $V(r)$ the Hamiltonian of the particle is \[\hat{H}=\frac{\hat{p}^2}{2m}+V(r)=-\frac{\hbar^2}{2m}\nabla^2+V(r)\] 
In spherical coordinates the Laplacian $\nabla^2$ is \[\nabla^2f=\frac{1}{r^2}\pdv{r}\qty(r^2\pdv{f}{r})+\frac{1}{r^2\sin\theta}\pdv{\theta}\qty(\sin\theta\pdv{f}{\theta})+\frac{1}{r^2\sin^2\theta}\pdv[2]{f}{\varphi}\] 
Angular momentum operator \[\hat{L}=\hat{r}\cp\hat{p}=i\hbar(\hat{r}\cp\nabla)\]
Hence Hamiltonian is \[\hat{H}=-\frac{\hbar^2}{2m}\frac{1}{r^2}\pdv{r}\qty(r^2\pdv{r})+\frac{\hat{L}^2}{2mr^2}+V(r)\]
The three components of angular momentum $\hat{L}\quad$$\hat{L}_x$,$\hat{L}_y$,$\hat{L}_z$ commute with Hamiltonian $\hat{H}$.
The three eigen value equations are 
\[\hat{H}\psi(r,\theta,\varphi)=E\psi(r,\theta,\varphi)\]
\[\hat{L}^2\psi(r,\theta,\varphi)=l(l+1)\hbar^2\psi(r,\theta,\varphi)\]
\[\hat{L}_z\psi(r,\theta,\varphi)=m\hbar\psi(r,\theta,\varphi)\]

Using separation of variables $\psi(r,\theta,\varphi)=R_{nl}\psi(r)Y^m_l\psi(\theta,\varphi)$.
The spherical harmonics are normalized according to the relation \[\int_{0}^{2\pi}\int_{0}^{\pi}Y^{m'}_{l}Y^{m'}_{l}\sin\theta\dd\theta\dd\varphi=\delta_{ll'}\delta_{mm'}\]
For radial functions, $\int_{0}^{\infty}r^2|R_{nl}^2(r)|\dd{r}=1$
\[\qty[-\frac{\hbar^2}{2m}\frac{1}{r^2}\pdv{r}\qty(r^2\pdv{r})+\frac{l(l+1)\hbar^2}{2mr^2}+V(r)]R_{nl}(r)=ER_{nl}(r)\]
Making the substitution $R_{nl}(r)=\frac{1}{r}U_{nl}(r)$, effective potential is $V_{eff}(r)=V(r)+\frac{l(l+1)\hbar^2}{2mr^2}$
\\
For the angular part, \[-i\hbar\pdv{\varphi}Y^m_l(\theta,\varphi)=m\hbar Y^m_l(\theta,\varphi)\]
\[\qty[\frac{1}{sin\theta}\pdv{\theta}\qty(\sin\theta\pdv{\theta})+\frac{1}{sin^2\theta}\pdv[2]{\varphi}]Y^m_l(\theta,\varphi)=l(l+1)\hbar^2 Y^m_l(\theta,\varphi)\]
\section{EM field}
Electric field $\vb{E}$ and magnetic field $\vb{B}$
are given by \[\vb{E}=-\grad{\phi}-\pdv{\vb{A}}{t}\qq{and}\vb{B}=\curl{\vb{A}}\]
When $\vb{E}$ and $\vb{B}$ are given, $\phi$ and $\vb{A}$ are not unique.
\[\phi'=\phi-\pdv{f(\vb{r},t)}{t}\qq{and}\vb{A'}=\vb{A}+\grad{f(\vb{r},t)}\]
\[\qq{Symmetric gauge} \vb{A}=-\frac{1}{2}\vb{r}\cp\vb{B}\]
\[\hat{H}=\frac{1}{2m}(\hat{p}-q\mathbf{A})\cdot(\hat{p}-q\mathbf{A})+ q\phi\]
\[\qq{probability density}\rho=\Psi^*(\vb{r},t)\Psi(\vb{r},t)\]
\[\qq{probability current density}\vb{s}=\frac{1}{2m}\qty[\frac{\hbar}{i}(\Psi^*(\vb{r},t)\grad{\Psi(\vb{r},t)}-\Psi(\vb{r},t)\grad{\Psi^*(\vb{r},t)})-2q\vb{A}\Psi^*(\vb{r},t)\Psi(\vb{r},t)]\]
\[\vb{s}=\frac{1}{2m}\qty[\frac{\hbar}{i}(\Psi^*(\vb{r},t)\grad{\Psi(\vb{r},t)}-\Psi(\vb{r},t)\grad{\Psi^*(\vb{r},t)})-2q\vb{A}\Psi^*(\vb{r},t)\Psi(\vb{r},t)]+\frac{\mu}{S}\curl{(\Psi^*(\vb{r},t)\hat{S}\Psi(\vb{r},t))}\]
\subsection*{equation of motion}
Use \[\vb{E}=-\grad{\phi}-\pdv{\vb{A}}{t}\qq{and}\vb{B}=\curl{\vb{A}}\]
and Hamiltonian \[\hat{H}=\frac{1}{2m}(\hat{p}-q\mathbf{A})\cdot(\hat{p}-q\mathbf{A})+ q\phi\]
and the Hamilton's equations of motion \[\dot{\vb{r}}=\pdv{H}{\vb{p}}\quad\dot{\vb{p}}=-\pdv{H}{\vb{r}}\]
to derive classical equation of motion \[m\vb{a}=q\vb{E}+q(\vb{v}\cp\vb{B})\]
\[\dot{\vb{r}}=\pdv{H}{\vb{p}}=\frac{1}{m}(\vb{p}-q\vb{A})=\vb{v}\]
\[\ddot{\vb{r}}=\dv{t}\qty(\frac{1}{m}(\vb{p}-q\vb{A}))=\frac{1}{m}(\dot{\vb{p}}-q\dot{\vb{A}})\]
\[\dot{\vb{p}}=-\pdv{H}{\vb{r}}=-\grad{H}=-\grad{\qty[\frac{1}{2m}(\hat{p}-q\mathbf{A})\cdot(\hat{p}-q\mathbf{A})+ q\phi]}\]
%\[\dot{\vb{p}}=-\grad{\qty[\frac{1}{2m}(\hat{p}-q\mathbf{A})\cdot(\hat{p}-q\mathbf{A})]}+ q\grad{\phi}}\]
\end{document}

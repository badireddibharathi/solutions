% !TEX TS-program = pdflatex
% !TEX encoding = UTF-8 Unicode

\documentclass[12pt]{article}

\usepackage[utf8]{inputenc} 

\usepackage{geometry}
\geometry{a4paper} 
\geometry{margin=0.25in} 
\geometry{portrait}
\usepackage{bm}
\usepackage{amsmath}
\usepackage{physics}
\usepackage{tikz} 

\title{greenwood}
\author{vijayabhaskar badireddi}

\begin{document}

\section*{Introduction}
Configuration space,
degrees of freedom.Generalized coordinates.constraints.
\[\pdv{(x_1,x_2,x_3,\ldots x_{3N})}{(q_1,q_2,q_3,\ldots q_{3N})}\neq 0\]
A dynamical system has $n$ generalized coordinates $q_1,q_2,\ldots,q_n$ and $k$ independent equations of constraint \[\phi_j(q_1,q_2,\ldots,q_n,t)=0\quad(j=1,2,\ldots,k)\]
\[\qq{Holonomic constraints}\]
\[\qq{scleronomic constraints}\]
\[\qq{rheonomic constraints}\]
non integrable 
\[\qq{non Holonomic constraints}\sum_{i=1}^{n}a_{ji}\dd{q_i}+a_{jt}\dd{t}=0\quad(j=1,2,\ldots,m)\]
\section*{Lagrange's equations}
\section*{applications}
\section*{Hamilton's equations}
\section*{Hamilton-Jacobi}
\section*{Canonical transformations}
\section*{relativity}
\paragraph{Lorentz transformations}
Two reference frames are in uniform relative motion with respect to each other. Frame $I$ is moving with respect to frame $I'$ with a constant velocity $V$ along the $x$ or $x'$ axis. The orgins of the frames coincide at time $t=0$ or $t'=0$. Since the speed of light has the same value $c$ in all directions with respect to each reference frame, a flash of light starting at the common origin at time $t=0$ or $t'=0$ will expand as a spherical wave front in both frames. The location of wave front is \[x^2+y^2+z^2=c^2t^2\qq{ in frame} I \] and \[ x'^2+y'^2+z'^2=c^2t'^2\qq{ in frame} I'\].
The transformations must be linear.
\[\qq{by symmetry}y'=y\qq{and}z'=z\]
\[x'=ax+bt\qq{and}t'=ex+ft\]
\newpage
Substitute in \[x^2-c^2t^2=x'^2-c^2t'^2\]
\[x^2-c^2t^2=(ax+bt)^2-c^2(ex+ft)^2\]
\[x^2-c^2t^2=(a^2-c^2e^2)x^2+(b^2-c^2f^2)t^2+2(ab-c^2ef)xt\]
By equating the coefficients of $x^2$,
\[a^2-c^2e^2=1\]
By equating the coefficients of $t^2$,
\[b^2-c^2f^2=-c^2\qq{hence}\]
By equating the coefficients of $xt$,
\[ab-c^2ef\]
Origin of frame $I'$ is denoted by $x'=0$ or $x=Vt$.
Using them in the equations of transformations,
\[x'=ax+bt\]
\end{document}

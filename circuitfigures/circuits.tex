% !TEX TS-program = pdflatex
% !TEX encoding = UTF-8 Unicode

\documentclass[12pt]{article}

\usepackage[utf8]{inputenc} 

\usepackage{geometry}
\geometry{a4paper} 
\geometry{margin=0.25in} 
\geometry{portrait}

\usepackage{amsmath}
\usepackage{physics}
\usepackage{circuitikz} 

\title{circuits}
\author{vijayabhaskar badireddi}

\begin{document}
\section*{circuits}

%\subsection*{resistances in series}

\begin{center}
\begin{circuitikz}[scale=1]
 \draw (0,0) node [anchor=east] {a} ;
 \draw (0,0) to [R,l=$R_1$,i=$i$] (2,0);
 \draw (2,0) to [R,l=$R_2$] (4,0);
 \draw (4,0) node [anchor=west] {b} ;
 \draw (6,0) node [anchor=east] {a} ;
 \draw (6,0) to [R,l=$R_1+R_2$,i_=$i$] (8,0);
 \draw (8,0) node [anchor=west] {b} ;
 \draw (2,-1) node [anchor=north] {resistances in series};
 \draw (5,0) node [] {$\equiv$};
 \end{circuitikz}
\end{center}
%\subsection*{resistances in parallel}

\begin{center}
\begin{circuitikz}[scale=1]
 \draw (0,0) node [anchor=east] {a} ;
 \draw (0,0) to (1,0);
 \draw (1,0) to (1,1);
 \draw (1,0) to (1,-1);
 \draw (3,1) to (3,0);
 \draw (3,-1) to (3,0);
 \draw (3,0) to (4,0);
 \draw (1,1) to [R,l=$R_1$,i_=$i_1$] (3,1);
 \draw (1,-1) to [R,l=$R_2$,i_=$i_2$] (3,-1);
 \draw (4,0) node [anchor=west] {b} ;
 \draw (6,0) node [anchor=east] {a} ;
 \draw (6,0) to [R,l=$\frac{R_1R_2}{R_1+R_2}$,i_=$i_1+i_2$] (8,0);
 \draw (8,0) to (9,0);
 \draw (9,0) node [anchor=west] {b} ;
 \draw (2,-2) node [anchor=north] {resistances in parallel};
 \end{circuitikz}
\end{center}
%\subsection*{capacitances in series}

\begin{center}
\begin{circuitikz}[scale=1]
 \draw (0,0) node [anchor=east] {a} ;
 \draw (0,0) to [C,l=$C_1$] (2,0);
 \draw (1,0) node [anchor=north east] {$+q$};
 \draw (1,0) node [anchor=north west] {$-q$};
 \draw (2,0) to [C,l=$C_2$] (4,0);
 \draw (3,0) node [anchor=north east] {$+q$};
 \draw (3,0) node [anchor=north west] {$-q$};
 \draw (4,0) node [anchor=west] {b} ;
 \draw (6,0) node [anchor=east] {a} ;
 \draw (6,0) to [C,l=$\frac{C_1C_2}{C_1+C_2}$] (8,0);
 \draw (7,0) node [anchor=north east] {$+q$};
 \draw (7,0) node [anchor=north west] {$-q$};
 \draw (8,0) node [anchor=west] {b} ;
 \draw (2,-1) node [anchor=north] {resistances in series};
 \draw (5,0) node [] {$\equiv$};
 \end{circuitikz}
\end{center}

%include{capparallel}
%\subsection*{charging RC circuit}

\begin{center}
\begin{circuitikz}[scale=1]
 \draw (0,0) node [anchor=east] {a} ;
 \draw (4,0) node [anchor=west] {b} ;
 \draw (0,0) to [battery1,l_=$\mathcal{E}$] (4,0);
 \draw (0,2) to [R,l=$R$] (2,2);
 \draw (0,0) to [short,i=$i$] (0,2);
 \draw (2,2) to [C,l=$C$] (4,2);
 \draw (3,2) node [anchor=north east] {$+q$};
 \draw (3,2) node [anchor=north west] {$-q$};
 \draw (4,2) to [short] (4,0);
 \draw (2,-1) node [anchor=north] {charging $RC$ circuit};
 \end{circuitikz}
\end{center}

\subsection*{discharging RC circuit}

\begin{center}
\begin{circuitikz}[scale=1]
 \draw (0,0) node [anchor=east] {a} ;
 \draw (4,0) node [anchor=west] {b} ;
 \draw (0,0) to [short,i=$i$] (4,0);
 \draw (0,2) to [R,l=$R$] (2,2);
 \draw (0,0) to [short] (0,2);
 \draw (2,2) to [C,l=$C$] (4,2);
 \draw (3,2) node [anchor=north east] {$+q$};
 \draw (3,2) node [anchor=north west] {$-q$};
 \draw (4,2) to [short] (4,0);
 \draw (2,-1) node [anchor=north] {discharging $RC$ circuit};
 \end{circuitikz}
 \end{center}
%% !TEX root = circuits.tex

\subsection*{AC circuit 1}

\begin{center}
\begin{circuitikz}[scale=1]
 \draw (0,0) node [anchor=east] {a} ;
 \draw (4,0) node [anchor=west] {b} ;
 \draw (0,0) to [vco,l_=$\mathcal{E}_0\sin\omega{t}$] (4,0);
 \draw (0,2) to [R,l=$R$] (2,2);
 \draw (0,0) to (0,2);
 \draw (2,2) to [C,l=$C$] (4,2);
% \draw (3,2) node [anchor=north east] {$+q$};
% \draw (3,2) node [anchor=north west] {$-q$};
 \draw (4,2) to (4,0);
 \draw (2,-2) node [anchor=north] {$RC$ circuit};
 \end{circuitikz}
\end{center}

\subsection*{AC circuit 2}

\begin{center}
\begin{circuitikz}[scale=1]
 \draw (0,0) node [anchor=east] {a} ;
 \draw (4,0) node [anchor=west] {b} ;
 \draw (0,0) to [vco,l_=$\mathcal{E}_0\sin\omega{t}$] (4,0);
 \draw (0,2) to [R,l=$R$] (2,2);
 \draw (0,0) to (0,2);
 \draw (2,2) to [L,l=$L$] (4,2);
% \draw (3,2) node [anchor=north east] {$+q$};
% \draw (3,2) node [anchor=north west] {$-q$};
 \draw (4,2) to (4,0);
 \draw (2,-2) node [anchor=north] {$RL$ circuit};
 \end{circuitikz}
\end{center}

\subsection*{AC circuit 3}

\begin{center}
\begin{circuitikz}[scale=1]
 \draw (0,0) node [anchor=east] {a} ;
 \draw (4,0) node [anchor=west] {b} ;
 \draw (0,0) to [vco,l_=$\mathcal{E}_0\sin\omega{t}$] (4,0);
 \draw (0,2) to [L,l=$L$] (2,2);
 \draw (0,0) to (0,2);
 \draw (2,2) to [C,l=$C$] (4,2);
% \draw (3,2) node [anchor=north east] {$+q$};
% \draw (3,2) node [anchor=north west] {$-q$};
 \draw (4,2) to (4,0);
 \draw (2,-2) node [anchor=north] {$LC$ circuit};
 \end{circuitikz}
\end{center}

\subsection*{AC circuit 4}

\begin{center}
\begin{circuitikz}[scale=1]
 \draw (0,0) node [anchor=east] {a} ;
 \draw (6,0) node [anchor=west] {b} ;
 \draw (0,0) to [vco,l_=$\mathcal{E}_0\sin\omega{t}$] (6,0);
 \draw (0,2) to [L,l=$L$] (2,2);
 \draw (2,2) to [R,l=$R$] (4,2);
 \draw (0,0) to (0,2);
 \draw (4,2) to [C,l=$C$] (6,2);
% \draw (3,2) node [anchor=north east] {$+q$};
% \draw (3,2) node [anchor=north west] {$-q$};
 \draw (6,2) to (6,0);
 \draw (3,-2) node [anchor=north] {$RLC$ circuit};
 \end{circuitikz}
\end{center}

\subsection*{transformers}

\begin{center}
\begin{circuitikz}[scale=1]

%\ctikzset{double bipoles/transformer/height=2} ;
\draw
 (0,0) node[transformer] (T) {}
 (T.A1) node[anchor=south] {A1}
 (T.A2) node[anchor=north] {A2}
 (T.B1) node[anchor=south] {B1}
 (T.B2) node[anchor=north] {B2}
 (T.base) node{K} ;
 \draw (-2,0) to [short] (T.A1) ;
  \draw (-2,-2) to [short] (T.A2) ;
 \draw (-2,0) to [vco,l_=$\mathcal{E}_0\sin\omega{t}$] (-2,-2) ;
\end{circuitikz}
\end{center}



\subsection*{RC circuit}
\begin{center}
\begin{circuitikz}

\draw (1,1) to [battery1,l_=$\mathcal{E}$] (9,1) ;
\draw (1,1) to [short] (1,4) to [short,i_=$i$] (2,4) ;
\draw (8,4) to [short] (9,4) to [short] (9,1);
\draw (2,4) to [short,i=$i_2$] (2,3) ;
\draw (2,4) to [short,i=$i_1$] (2,5) ;
\draw (8,3) to [short] (8,4) ;
\draw (8,4) to [short] (8,5) ;
\draw (2,5) to [R,l=$R_1$] (4,5) ;
\draw (2,3) to [R,l_=$R_2$] (4,3) ;
\draw (4,3) to [short] (4,5) ;
\draw (6,5) to [C,l=$C_1$] (8,5) ;
\draw (7,5) node [anchor=south east] {$+q_1$} ; 
\draw (7,5) node [anchor=south west] {$-q_1$} ; 
\draw (6,3) to [C,l_=$C_2$] (8,3) ;
\draw (7,3) node [anchor=south east] {$+q_2$} ; 
\draw (7,3) node [anchor=south west] {$-q_2$} ; 
\draw (4,4) to [short] (6,4) ;
\draw (6,3) to [short] (6,5) ;
\end{circuitikz}
\end{center}

\subsection*{large plates}
\begin{center}
\begin{circuitikz}[scale=1]

\draw (0,0) rectangle ++(0.2,3) ;
\draw (1,0) rectangle ++(0.2,3) ;
\draw (2,0) rectangle ++(0.2,3) ;
\draw (3,0) rectangle ++(0.2,3) ;
\draw (0.1,3) to [short] ++(0,0.5) to ++(2,0) to ++(0,-0.5) ;
\draw (1.1,0) to [short] ++(0,-0.5) to [battery1,l_=$\mathcal{E}$] ++(2,0) to ++(0,0.5) ;
\end{circuitikz}
\end{center}

\subsection*{Force on dielectric}
\begin{center}
\begin{circuitikz}[scale=1]

\draw (0,2) to [battery1,l_=$\mathcal{E}$] (0,0) ;

\draw (0,0) to [short] (0.5,0) ;
\draw (0,2) to [short] (0.5,2) ;
\draw [fill,color=gray] (0.5,-0.1) rectangle ++(4,0.2) ; 
\draw [fill,color=gray] (0.5,1.9) rectangle ++(4,0.2) ; 
\draw (2.5,0.1) rectangle ++(4,1.8) ; 
\draw [-{latex}] (3,2.5) -- ++(1.5,0) ;
\draw [-{latex}] (2,2.5) -- ++(-1.5,0) ;
\draw [-{latex}] (5,-0.5) -- ++(1.5,0) ;
\draw [-{latex}] (4,-0.5) -- ++(-1.5,0) ;
\draw (2.5,2.5) node {$a$};
\draw (4.5,-0.5) node {$a$};
\end{circuitikz}
\end{center}

\subsection*{gausslawcylinder}
\begin{center}
\begin{circuitikz}[scale=1]

\draw [color=gray] (0,6) -- (0,-1) ;
\draw (0,0) ellipse [x radius=2,y radius=0.5] ;
\draw (0,5) ellipse [x radius=2,y radius=0.5] ;
\draw (2,0) -- (2,5) ;
\draw (-2,0) -- (-2,5) ;

\draw [very thick] (0,-1) ellipse [x radius=1,y radius=0.25] ;
\draw [very thick] (0,6) ellipse [x radius=1,y radius=0.25] ;
\draw [very thick] (1,-1) -- (1,6) ;
\draw [very thick] (-1,-1) -- (-1,6) ;

\end{circuitikz}
\end{center}

\end{document}

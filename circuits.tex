% !TEX TS-program = pdflatex
% !TEX encoding = UTF-8 Unicode

\documentclass[12pt]{article}

\usepackage[utf8]{inputenc} 

\usepackage{geometry}
\geometry{a4paper} 
\geometry{margin=0.25in} 
\geometry{portrait}

\usepackage{amsmath}
\usepackage{physics}
\usepackage{circuitikz} 

\title{circuits}
\author{vijayabhaskar badireddi}

\begin{document}
\section*{circuits}
\subsection*{resistances in series}

\begin{center}
\begin{circuitikz}[scale=1]
 \draw (0,0) node [anchor=east] {a} ;
 \draw (0,0) to [R,l=$R_1$,i=$i$] (2,0);
 \draw (2,0) to [R,l=$R_2$] (4,0);
 \draw (4,0) node [anchor=west] {b} ;
 \draw (6,0) node [anchor=east] {a} ;
 \draw (6,0) to [R,l=$R_1+R_2$,i_=$i$] (8,0);
 \draw (8,0) node [anchor=west] {b} ;
 \draw (2,-1) node [anchor=north] {resistances in series};
 \draw (5,0) node [] {$\equiv$};
 \end{circuitikz}
\end{center}

\subsection*{resistances in parallel}

\begin{center}
\begin{circuitikz}[scale=1]
 \draw (0,0) node [anchor=east] {a} ;
 \draw (0,0) to (1,0);
 \draw (1,0) to (1,1);
 \draw (1,0) to (1,-1);
 \draw (3,1) to (3,0);
 \draw (3,-1) to (3,0);
 \draw (3,0) to (4,0);
 \draw (1,1) to [R,l=$R_1$,i_=$i_1$] (3,1);
 \draw (1,-1) to [R,l=$R_2$,i_=$i_2$] (3,-1);
 \draw (4,0) node [anchor=west] {b} ;
 \draw (6,0) node [anchor=east] {a} ;
 \draw (6,0) to [R,l=$\frac{R_1R_2}{R_1+R_2}$,i_=$i_1+i_2$] (8,0);
 \draw (8,0) to (9,0);
 \draw (9,0) node [anchor=west] {b} ;
 \draw (2,-2) node [anchor=north] {resistances in parallel};
 \end{circuitikz}
\end{center}

\subsection*{capacitances in series}

\begin{center}
\begin{circuitikz}[scale=1]
 \draw (0,0) node [anchor=east] {a} ;
 \draw (0,0) to [C,l=$C_1$] (2,0);
 \draw (1,0) node [anchor=north east] {$+q$};
 \draw (1,0) node [anchor=north west] {$-q$};
 \draw (2,0) to [C,l=$C_2$] (4,0);
 \draw (3,0) node [anchor=north east] {$+q$};
 \draw (3,0) node [anchor=north west] {$-q$};
 \draw (4,0) node [anchor=west] {b} ;
 \draw (6,0) node [anchor=east] {a} ;
 \draw (6,0) to [C,l=$\frac{C_1C_2}{C_1+C_2}$] (8,0);
 \draw (7,0) node [anchor=north east] {$+q$};
 \draw (7,0) node [anchor=north west] {$-q$};
 \draw (8,0) node [anchor=west] {b} ;
 \draw (2,-1) node [anchor=north] {resistances in series};
 \draw (5,0) node [] {$\equiv$};
 \end{circuitikz}
\end{center}

\subsection*{capacitances in parallel}

\begin{center}
\begin{circuitikz}[scale=1]
 \draw (0,0) node [anchor=east] {a} ;
 \draw (0,0) to (1,0);
 \draw (1,0) to (1,1);
 \draw (1,0) to (1,-1);
 \draw (3,1) to (3,0);
 \draw (3,-1) to (3,0);
 \draw (3,0) to (4,0);
 \draw (1,1) to [C,l=$C_1$] (3,1);
 \draw (2,1) node [anchor=north east] {$+q_1$};
 \draw (2,1) node [anchor=north west] {$-q_1$};
 \draw (1,-1) to [C,l=$C_2$] (3,-1);
 \draw (2,-1) node [anchor=north east] {$+q_2$};
 \draw (2,-1) node [anchor=north west] {$-q_2$};
 \draw (4,0) node [anchor=west] {b} ;
 \draw (5,0) node [] {$\equiv$};
 \draw (6,0) node [anchor=east] {a} ;
 \draw (6,0) to [C,l=$C_1+C_2$] (8,0);
 \draw (7,0) node [anchor=north east] {$+q$};
 \draw (7,0) node [anchor=north west] {$-q$};
 \draw (8,0) node [anchor=west] {b} ;
 \draw (2,-2) node [anchor=north] {resistances in parallel};
 \end{circuitikz}
\end{center}

\subsection*{charging RC circuit}

\begin{center}
\begin{circuitikz}[scale=1]
 \draw (0,0) node [anchor=east] {a} ;
 \draw (4,0) node [anchor=west] {b} ;
 \draw (0,0) to [battery1,l_=$\mathcal{E}$] (4,0);
 \draw (0,2) to [R,l=$R$,i=$i$] (2,2);
 \draw (0,0) to (0,2);
 \draw (2,2) to [C,l=$C$] (4,2);
 \draw (3,2) node [anchor=north east] {$+q$};
 \draw (3,2) node [anchor=north west] {$-q$};
 \draw (4,2) to (4,0);
 \draw (2,-1) node [anchor=north] {charging $RC$ circuit};
 \end{circuitikz}
\end{center}

\subsection*{discharging RC circuit}

\begin{center}
\begin{circuitikz}[scale=1]
 \draw (0,0) node [anchor=east] {a} ;
 \draw (4,0) node [anchor=west] {b} ;
 \draw (0,0) to (4,0);
 \draw (0,2) to [R,l=$R$,i<=$i$] (2,2);
 \draw (0,0) to (0,2);
 \draw (2,2) to [C,l=$C$] (4,2);
 \draw (3,2) node [anchor=north east] {$+q$};
 \draw (3,2) node [anchor=north west] {$-q$};
 \draw (4,2) to (4,0);
 \draw (2,-1) node [anchor=north] {discharging $RC$ circuit};
 \end{circuitikz}
\end{center}

\end{document}

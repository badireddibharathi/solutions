% !TEX TS-program = pdflatex
% !TEX encoding = UTF-8 Unicode
\documentclass[12pt]{article} 

\usepackage[utf8]{inputenc} 
\usepackage{geometry} 
\geometry{a4paper} 
\geometry{margin=0.25in} 
\geometry{portrait} 

\usepackage{tikz} 

\usepackage{amsmath} 
\usepackage{physics} 
\title{Fresnel}
\author{vijayabhaskar badireddi} 
\date{} 

\begin{document}
\section*{Fresnel's equations}
In  the region where there is no charge and current, maxwell's equations are 
\begin{align}
\div{\vb{E}}&=0\\
\div{\vb{B}}&=0\\
\curl{\vb{E}}&=-\pdv{\vb{B}}{t}\\
\curl{\vb{B}}&=\mu_0\epsilon_0\pdv{\vb{E}}{t}
\end{align}
Applying curl to $(3)$ and $(4)$ 
\[\curl{\qty(\curl{\vb{E}})}=\grad{\qty(\div{\vb{E}})}-\laplacian{\vb{E}}=\curl{\qty(-\pdv{\vb{B}}{t})}=-\pdv{t}\qty(\curl{\vb{B}})=-\mu_)\epsilon_0\pdv[2]{\vb{E}}{t}\]

\[\curl{\qty(\curl{\vb{B}})}=\grad{\qty(\div{\vb{B}})}-\laplacian{\vb{B}}=\curl{\qty(\mu_0\epsilon_0\pdv{\vb{E}}{t})}=\mu_0\epsilon_0\pdv{t}\qty(\curl{\vb{E}})=-\mu_0\epsilon_0\pdv[2]{\vb{B}}{t}\]
Using $(1)$ and $(2)$ \\
\[\boxed{\laplacian{\vb{E}}=\mu_0\epsilon_0\pdv[2]{\vb{E}}{t}\quad\laplacian{\vb{B}}=\mu_0\epsilon_0\pdv[2]{\vb{B}}{t}}\]
If the propagation vector is $\vb{k}$, direction of propagation is in the same direction.
The electric and magnetic fields are given by
\[\boxed{\tilde{\vb{E}}(\vb{r},t)=\tilde{E}_0e^{i(\vb{k}\cdot\vb{r}-\omega t)}\vu{n}}\]
\[\boxed{\tilde{\vb{B}}(\vb{r},t)=\frac{1}{c}\tilde{E}_0e^{i(\vb{k}\cdot\vb{r}-\omega t)}(\vu{k}\cp\vu{n})=\frac{1}{c}\vu{k}\cp\tilde{\vb{E}}}\]
Let the $xy$ plane be the boundary between two linear media. A plane wave of frequency $\omega$ is travelling in the $z$ direction and polarized in the $x$ direction.\\
Incident waves are \[\tilde{\vb{E}}_I(z,t)=\tilde{E}_{0I}e^{i(k_1z-\omega t)}\vu{x}\]
\[\tilde{\vb{B}}_I(z,t)=\frac{1}{v_1}\tilde{E}_{0I}e^{i(k_1z-\omega t)}\vu{y}\]
The reflected waves are  \[\tilde{\vb{E}}_R(z,t)=\tilde{E}_{0R}e^{i(-k_1z-\omega t)}\vu{x}\]
\[\tilde{\vb{B}}_R(z,t)=-\frac{1}{v_1}\tilde{E}_{0I}e^{i(-k_1z-\omega t)}\vu{y}\]
The transmitted waves are  \[\tilde{\vb{E}}_T(z,t)=\tilde{E}_{0T}e^{i(k_2z-\omega t)}\vu{x}\]
\[\tilde{\vb{B}}_T(z,t)=\frac{1}{v_2}\tilde{E}_{0T}e^{i(k_2z-\omega t)}\vu{y}\]
Using boundary conditions,\\
\[\tilde{E}_{0I}+\tilde{E}_{0R}=\tilde{E}_{0T}\]
\[\frac{1}{\mu_1}\qty(\frac{1}{v_1}\tilde{E}_{0I}-\frac{1}{v_1}\tilde{E}_{0R})=\frac{1}{\mu_2}\qty(\frac{1}{v_2}\tilde{E}_{0T})\]
\[\tilde{E}_{0I}-\tilde{E}_{0R}=\beta\tilde{E}_{0T}\qq{ where }\beta=\frac{\mu_1v_1}{\mu_2v_2}\]
\[\tilde{E}_{0R}=\qty(\frac{1-\beta}{1+\beta})\tilde{E}_{0I}\quad\tilde{E}_{0T}=\qty(\frac{2}{1+\beta})\tilde{E}_{0I}\]
\section*{Oblique incidence}
\[\tilde{\vb{E}}_I(\vb{r},t)=\tilde{\vb{E}}_{0I}e^{i(\vb{k}_I\cdot\vb{r}-\omega t)}\quad \tilde{\vb{B}}_I(\vb{r},t)=\frac{1}{v_1}\qty(\vu{k}_I\cp\tilde{\vb{E}}_I)\]
\[\tilde{\vb{E}}_R(\vb{r},t)=\tilde{\vb{E}}_{0R}e^{i(\vb{k}_R\cdot\vb{r}-\omega t)}\quad \tilde{\vb{B}}_R(\vb{r},t)=\frac{1}{v_1}\qty(\vu{k}_R\cp\tilde{\vb{E}}_R)\]
\[\tilde{\vb{E}}_T(\vb{r},t)=\tilde{\vb{E}}_{0T}e^{i(\vb{k}_T\cdot\vb{r}-\omega t)}\quad \tilde{\vb{B}}_T(\vb{r},t)=\frac{1}{v_2}\qty(\vu{k}_T\cp\tilde{\vb{E}}_T)\]
\[k_Iv_1=k_Rv_1=k_Tv_2=\omega\]
Using boundary conditions and matching phases,
\[\vb{k}_I\vdot\vb{r}=\vb{k}_R\vdot\vb{r}=\vb{k}_T\vdot\vb{r}\qq{where} z=0\]
\[x(k_I)_x+y(k_I)_y=x(k_R)_x+y(k_R)_y=x(k_T)_x+y(k_T)_y\]
\[(k_I)_x=(k_R)_x=(k_T)_x\]
\[(k_I)_y=(k_R)_y=(k_T)_y\]
\[k_I\sin\theta_I=k_R\sin\theta_R=k_T\sin\theta_T\]
\[\theta_I=\theta_R\]
\[\frac{\sin\theta_T}{\sin\theta_I}=\frac{v_2}{v_1}=\frac{n_1}{n_2}\]
Boundary conditions are 
\begin{align*}
\epsilon_1\qty(\tilde{\vb{E}}_{0I}+\tilde{\vb{E}}_{0R})_z&=\epsilon_2\qty(\tilde{\vb{E}}_{0T})_z\\
\qty(\tilde{\vb{B}}_{0I}+\tilde{\vb{B}}_{0R})_z&=\qty(\tilde{\vb{B}}_{0T})_z\\
\qty(\tilde{\vb{E}}_{0I}+\tilde{\vb{E}}_{0R})_{x,y}&=\qty(\tilde{\vb{E}}_{0T})_{x,y}\\
\frac{1}{\mu_1}\qty(\tilde{\vb{B}}_{0I}+\tilde{\vb{B}}_{0R})_{x,y}&=\frac{1}{\mu_2}\qty(\tilde{\vb{B}}_{0T})_{x,y}
\end{align*}
If the polarization of incident wave is parallelto the plane of incidence,
\begin{align*}
\epsilon_1(-\tilde{E}_{0I}\sin\theta_I+\tilde{E}_{0R}\sin\theta_R)&=\epsilon_2(-\tilde{E}_{0T}\sin\theta_T)\\
\tilde{E}_{0I}\cos\theta_I+\tilde{E}_{0R}\cos\theta_R&=
\tilde{E}_{0T}\cos\theta_T\\
\frac{1}{\mu_1v_1}(\tilde{E}_{0I}-\tilde{E}_{0R})&=\frac{1}{\mu_2v_2}\tilde{E}_{0T}
\end{align*}
\[(\tilde{E}_{0I}-\tilde{E}_{0R})=\frac{\mu_1v_1}{\mu_2v_2}\tilde{E}_{0T}=\beta\tilde{E}_{0T}\]
\[(\tilde{E}_{0I}+\tilde{E}_{0R})=\alpha\tilde{E}_{0T}\qq{where}\alpha=\frac{\cos\theta_T}{\cos\theta_I}\]
\[\boxed{\tilde{E}_{0R}=\qty(\frac{\alpha-\beta}{\alpha+\beta})\tilde{E}_{0I}\quad\tilde{E}_{0T}=\qty(\frac{2}{\alpha+\beta})\tilde{E}_{0I}}\]
\end{document}

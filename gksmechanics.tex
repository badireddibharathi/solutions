% !TEX TS-program = pdflatex
% !TEX encoding = UTF-8 Unicode

\documentclass[12pt]{article}

\usepackage[utf8]{inputenc} 

\usepackage{geometry}
\geometry{a4paper} 
\geometry{margin=0.25in} 
\geometry{portrait}

\usepackage{amsmath}
\usepackage{physics}
\usepackage{tikz} 
\usepackage{bm}


\title{gksmechanics}
\author{vijayabhaskar badireddi}
%\date{} 

\begin{document}
%\maketitle

\section*{Introduction}

\section*{Lagrangian mechanics}
\subsection*{calculus of variations}
\[I=\int f(y,\dot{y},x)\dd{x}\]
The path for which $I$ is an extremum,$\delta{I}=0$ is given by \[\pdv{f}{y}-\dv{t}\qty(\pdv{f}{\dot{y}})=0\]
\subsection*{Hamilton's variational principle}
The motion of a dynamical system from time $t_1$ to $t_2$ is such that the intergral $\int_{t_1}^{t_2}(T(q_i,\dot{q}_i)-V(q_i))\dd{t}$ is an extremum.\\
Denoting $L=T-V$, \[\dv{t}\qty(\pdv{L}{\dot{q}_i})-\pdv{L}{q_i}=0\]
\subsection*{D'alembert's principle}
\[\dv{t}\qty(\pdv{T}{\dot{q}_i})-\pdv{T}{q_i}=Q_i\]
\subsection*{Non-Holonomic systems}
If Non-Holonomic constraints are expressed as $\sum_ka_{lk}\dd{q_k}+a_{lt}\dd{t}=0\qq{where}l=1,2,\ldots,m$ use Lagrange's method of undetermined multipliers.
\[\pdv{L}{q_k}-\dv{t}\qty(\pdv{L}{\dot{q}_k})+\sum_l\lambda_la_{lk}=0\]

\section*{Hamiltonian mechanics}
Generalized momentum $p_i=\pdv{L}{q_i}$, Phase space\\
Hamiltonian $H=\sum_ip_i\dot{q}_i-L$\\
Hamilton's canonical equations of motion
\begin{align*}
\dot{q}_i&=\pdv{H}{p_i}\\
\dot{p}_i&=-\pdv{H}{q_i}\\
-\pdv{L}{t}&=\pdv{H}{t}
\end{align*}
When is Hamiltonian a constant of motion ? When is Hamiltonian equal to energy ?\\
Hamilton's equations of motion from Variational principle.
\[\delta{I}=\delta\int_{t_1}^{t_2}L\dd{t}=0\]
Since $H=\sum_ip_i\dot{q}_i-L$,
\[\delta{I}=\delta\int_{t_1}^{t_2}\left[\sum_ip_i\dot{q}_i-H\right]\dd{t}=\delta\sum_i\int_{t_1}^{t_2}p_i\dd{q}_i-\delta\int_{t_1}^{t_2}H\dd{t}=0\]
\[\delta{I}=\dd{\alpha}\vdot\pdv{I}{\alpha}=\dd{\alpha}\vdot\pdv{\alpha}\int_{t_1}^{t_2}\left[\sum_ip_i\dot{q}_i-H\right]\dd{t}=\]
\[\delta{I}=\dd{\alpha}\int_{t_1}^{t_2}\sum_i\left(\pdv{p_i}{\alpha}\dot{q}_i+p_i\pdv{\dot{q}_i}{\alpha}-\pdv{H}{q_i}\pdv{q_i}{\alpha}-\pdv{H}{p_i}\pdv{p_i}{\alpha}-\pdv{H}{t}\pdv{t}{\alpha}\right)\dd{t}=0\]
Since $\pdv{t}{\alpha}=0$
\[\delta{I}=\dd{\alpha}\int_{t_1}^{t_2}\sum_i\left(\pdv{p_i}{\alpha}\dot{q}_i+p_i\pdv{\dot{q}_i}{\alpha}-\pdv{H}{q_i}\pdv{q_i}{\alpha}-\pdv{H}{p_i}\pdv{p_i}{\alpha}\right)\dd{t}=0\]
\[\qq{since}\int_{t_1}^{t_2}p_i\pdv{\dot{q}_i}{\alpha}\dd{t}=-\int_{t_1}^{t_2}p_i\pdv{q_i}{\alpha}\dd{t}\]
\[\delta{I}=\dd{\alpha}\int_{t_1}^{t_2}\sum_i\left(\pdv{p_i}{\alpha}\dot{q}_i-p_i\pdv{q_i}{\alpha}-\pdv{H}{q_i}\pdv{q_i}{\alpha}-\pdv{H}{p_i}\pdv{p_i}{\alpha}\right)\dd{t}=0\]

\[\delta{I}=\int_{t_1}^{t_2}\sum_i\left(\delta{p_i}\dot{q}_i-p_i\delta{q_i}-\pdv{H}{q_i}\delta{q_i}-\pdv{H}{p_i}\delta{p_i}\right)\dd{t}=0\]

\[\delta{I}=\int_{t_1}^{t_2}\sum_i\left(\delta{p_i}\left(\dot{q}_i-\pdv{H}{p_i}\right)-\delta{q_i}\left(p_i+\pdv{H}{q_i}\right)\right)\dd{t}=0\]
Hence
\[\dot{q}_i=\pdv{H}{p_i}\]
\[\dot{p}_i=-\pdv{H}{q_i}\] 
Principle of least action.
\[\qq{action} A=\int_{t_1}^{t_2}\sum_{i}p_i\dot{q}_i\dd{t}\]
\[\Delta\int_{t_1}^{t_2}\sum_{i}p_i\dot{q}_i\dd{t}=0\]
\[\Delta{f}=\delta{f}+\pdv{f}{t}\Delta{t}\]
\subsection*{Poisson brackets}
\[\dv{F}{t}=\pdv{F}{t}+\sum_{i}\qty(\pdv{F}{q_i}\pdv{H}{p_i}-\pdv{F}{p_i}\pdv{H}{q_i})=\pdv{F}{t}+\pb{F}{H}\]
\[\pb{X}{Y}=-\pb{Y}{X}\]
\[\pb{X}{X}=0\]
\[\pb{X}{Y+Z}=\pb{X}{Y}+\pb{X}{Z}\]
\[\pb{X}{YZ}=Y\pb{X}{Z}+\pb{X}{Y}Z\]
\[\pb{q_i}{p_j}=\delta_{ij}\]
Poisson brackets are invariant under a canonical transformation.\\
\[\dot{q}_i=\pb{q_i}{H}\quad\dot{p}_i=-\pb{p_i}{H}\]
\[\qq{Jacobi's identity}\pb{X}{\pb{Y}{Z}}+\pb{Z}{\pb{X}{Y}}+\pb{Y}{\pb{Z}{X}}=0\]
If $X$ and $Y$ are constants of motion, then their Poisson bracket $\pb{X}{Y}$ is also a constant of motion.
\[\pb{X}{H}=0\qq{and}\pb{Y}{H}=0\implies\pb{\pb{X}{Y}}{H}=0\]
\section*{Central forces}
Two body problem can be reduced to a one body problem. Two particles of masses $m_1$ and $m_2$ are located at $\vb{r}_1$ and $\vb{r}_2$ respectively. Their centre of mass is located at $\vb{r}_c=\frac{m_1\vb{r}_1+m_2\vb{r}_2}{m_1+m_2}$. Their difference vector is $\vb{r}=\vb{r}_1-\vb{r}_2$. Their position vectors relative to centre of mass are \[\vb{r}_{1c}=\vb{r}_{1}-\vb{r}_{c}=\vb{r}_{1}-\frac{m_1\vb{r}_1+m_2\vb{r}_2}{m_1+m_2}=\frac{m_2(\vb{r}_1-\vb{r}_2)}{m_1+m_2}=\frac{m_2}{m_1+m_2}\vb{r}\] and 
\[\vb{r}_{2c}=\vb{r}_{2}-\vb{r}_{c}=\vb{r}_{2}-\frac{m_1\vb{r}_1+m_2\vb{r}_2}{m_1+m_2}=\frac{m_1(\vb{r}_2-\vb{r}_1)}{m_1+m_2}=-\frac{m_1}{m_1+m_2}\vb{r}\]
The kinetic energy \[\frac{1}{2}m_1\dot{\vb{r}}_1^2+\frac{1}{2}m_2\dot{\vb{r}}_2^2=\frac{1}{2}m_1\qty(\dot{\vb{r}}_{c}+\frac{m_2}{m_1+m_2}\dot{\vb{r}})^2+\frac{1}{2}m_2\qty(\dot{\vb{r}}_{c}-\frac{m_1}{m_1+m_2}\dot{\vb{r}})^2\]
\[=\frac{1}{2}(m_1+m_2)\dot{\vb{r}}_{c}^2+\frac{1}{2}\qty(\frac{m_1m_2}{m_1+m_2})\dot{\vb{r}}^2=\frac{1}{2}M\dot{\vb{r}}_{c}^2+\frac{1}{2}\mu\dot{\vb{r}}^2\]
The potential energy \[V=-\frac{Gm_1m_2}{|\vb{r}_1-\vb{r}_2|}=-\frac{Gm_1m_2}{|\vb{r}|}\]
Hence Lagrangian of the system is \[L=T-V=\frac{1}{2}M\dot{r}_c^2+\frac{1}{2}\mu \dot{r}^2+\frac{Gm_1m_2}{r}\]
Equation of motion for $\vb{r}_c$ is \[\dv{t}\qty(\pdv{L}{\dot{r}_c})-\pdv{L}{r_c}=M\ddot{r}_c=0\]
Equation of motion for $\vb{r}$ is \[\dv{t}\qty(\pdv{L}{\dot{r}})-\pdv{L}{r}=\mu\ddot{r}+\frac{Gm_1m_2}{r^2}=\mu\ddot{r}+\frac{G(m_1+m_2)\mu}{r^2}=0\]
\[\ddot{r}+\frac{GM}{r^2}=0\]
Since the gravitational force \[\vb{F}=-\frac{Gm_1m_2\vb{r}}{|\vb{r}|^3}\]
satisfies \[\curl{\vb{F}}=0\]
it is conservative and energy is conserved.
Since \[\vb{N}=\dv{\vb{L}}{t}=\vb{r}\cp\vb{F}=\vb{r}\cp-\frac{Gm_1m_2\vb{r}}{|\vb{r}|^3}=0\]
angular momentum of the system $\vb{l}$ is conserved.\\
Lagrangian is \[L=\frac{1}{2}m(\dot{r}^2+r^2\dot{\theta}^2)-V(r)\]
Lagrangian equation for $\theta$ is \[\dv{t}\qty(\pdv{L}{\dot{\theta}})-\pdv{L}{\theta}=0\]
\[\dv{t}\qty(mr^2\dot{\theta})=0\]
\section*{Relative coordinates}

\[\qty(\dv{\vb{r}}{t})_{un primed}=\qty(\dv{\vb{r}}{t})_{primed}+\bm{\omega}\cp\vb{r}\]
\[\dv{\vb{r}}{t}=\dv{'\vb{r}}{t}+\bm{\omega}\cp\vb{r}\]
\[\dv{t}\qty(\dv{\vb{r}}{t})=\qty(\dv{'}{t}+\bm{\omega}\cp)\qty(\dv{'\vb{r}}{t}+\bm{\omega}\cp\vb{r})\]
\[\dv[2]{\vb{r}}{t}=\dv[2]{\vb{r}}{t}+2\bm{\omega}\cp\dv{'\vb{r}}{t}+\bm{\omega}\cp(\bm{\omega}\cp\vb{r})+\dv{'\bm{\omega}}{t}\cp\vb{r}\]

\section*{Rigid body motion}
\subsection*{Euler angles}
The space set of axes are $(x,y,z)$ with unit vectors $(\vu{x},\vu{y},\vu{z})$. The body set of axes are $(x',y',z')$ with unit vectors $(\vu{x}',\vu{y}',\vu{z}')$.\\
First rotation is about the space $z$ axis through an angle $\phi$.
\[
\begin{pmatrix}
\vu{x}_1\\ \vu{y}_1\\ \vu{z}_1
\end{pmatrix}
=
\begin{pmatrix}
\cos\phi& \sin\phi& 0\\
-\sin\phi& \cos\phi& 0\\
0& 0& 1
\end{pmatrix}
\begin{pmatrix}
\vu{x}\\ \vu{y}\\ \vu{z}
\end{pmatrix}
\]
The second rotation is about $x_1$ axis through an angle $\theta$ till $z_1$ axis coincides with $z'$ axis.
\[
\begin{pmatrix}
\vu{x}_2\\ \vu{y}_2\\ \vu{z}'
\end{pmatrix}
=
\begin{pmatrix}
1& 0& 0 \\
0& \cos\theta& \sin\theta\\
0& -\sin\theta& \cos\theta
\end{pmatrix}
\begin{pmatrix}
\vu{x}_1\\ \vu{y}_1\\ \vu{z}_1
\end{pmatrix}
\]
The third rotation is about $z'$ axis through an angle $\psi$ till $x_2$ axis coincides with $x'$ axis and $y_2$ axis coincides with $y'$ axis.
\[
\begin{pmatrix}
\vu{x}'\\ \vu{y}'\\ \vu{z}'
\end{pmatrix}
=
\begin{pmatrix}
\cos\psi& \sin\psi& 0 \\
-\sin\psi& \cos\psi& 0 \\
0& 0& 1
\end{pmatrix}
\begin{pmatrix}
\vu{x}_2\\ \vu{y}_2\\ \vu{z}'
\end{pmatrix}
\]

\[
\begin{pmatrix}
\vu{x}'\\ \vu{y}'\\ \vu{z}'
\end{pmatrix}
=
\begin{pmatrix}
\cos\psi& \sin\psi& 0 \\
-\sin\psi& \cos\psi& 0 \\
0& 0& 1
\end{pmatrix}
\begin{pmatrix}
1& 0& 0 \\
0& \cos\theta& \sin\theta\\
0& -\sin\theta& \cos\theta
\end{pmatrix}
\begin{pmatrix}
\cos\phi& \sin\phi& 0\\
-\sin\phi& \cos\phi& 0\\
0& 0& 1
\end{pmatrix}
\begin{pmatrix}
\vu{x}\\ \vu{y}\\ \vu{z}
\end{pmatrix}
\]
\[\qty(\dv{\vb{A}}{t})_{space}=\qty(\dv{\vb{A}}{t})_{body}+\bm{\omega}\cp\vb{A}\]


\section*{Relativity}

\[x^2-c^2t^2=x'^2-c^2t'^2\]
Lorenz transformations are orthogonal transformations in four dimensional space.
\[a_{\mu\nu}a_{\lambda\nu}=\delta_{\mu\lambda}\]

\[
\begin{pmatrix}
x_1'\\ x_2'\\ x_3'\\ x_4'
\end{pmatrix}
=
\begin{pmatrix}
1& 0& 0& 0\\ 0& 1& 0& 0\\ 0& 0& \gamma& i\gamma\beta\\ 0& 0& -i\gamma\beta& \gamma
\end{pmatrix}
\begin{pmatrix}
x_1\\ x_2\\ x_3\\ x_4
\end{pmatrix}
\]

\section*{Small oscilltions}
stable and unstable equilibrium

\end{document}

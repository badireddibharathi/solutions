% !TEX TS-program = pdflatex
% !TEX encoding = UTF-8 Unicode
	
\documentclass[12pt]{article} 

\usepackage[utf8]{inputenc} 
\usepackage{geometry} 
\geometry{a4paper} 
\geometry{margin=0.25in} 
\geometry{portrait} 

\usepackage{tikz} 

\usepackage{amsmath} 
\usepackage{physics} 
\title{solution}
\author{vijayabhaskar badireddi} 
\date{} 

\begin{document}
\maketitle
\[ (x,y)=x^{\dag}y \]
\[ x^{\dag}=(x^*)^T=(x^T)^* \]
\[ \left(A^T\right)^T=A \]
\[ \left(A^*\right)^*=A \]
\[ \left(A^\dag\right)^\dag=A \]
\[ \left(AB\right)^T=B^TA^T \]
\[ \left(AB\right)^*=A^*B^* \]
\[ \left(AB\right)^\dag=B^\dag A^\dag
\]
\section*{a)}
\[
(x,Ay)=x^\dag Ay=x^\dag \left(A^\dag \right )^\dag y=\left(A^{\dag}x \right )^\dag y=(A^\dag x,y)
\]
\section*{b)}
\[
Av=\lambda v \]\[ A^\dag=A
\]
\[
\left(Av\right)^\dag=\left(\lambda v\right)^\dag=v^\dag A^\dag=\lambda^*v^\dag
\]
\[
\left(v,Av \right )=v^\dag Av=\lambda v^\dag v
\]
\[
\left(v,Av \right )=v^\dag Av=v^\dag A^\dag v=\lambda^* v^\dag v
\]

Hence \[ \left(v,Av \right )=\lambda^* v^\dag v=\lambda v^\dag v
\]
\[
\lambda^*=\lambda \]\[ \lambda is real.
\]
\section*{c)}
\[
A^\dag=A
\]
\[
Av_1=\lambda_1v_1, Av_2=\lambda_2v_2 and \lambda_1\ne\lambda_2
\]
\[
\left(v_1,Av_2 \right )=\left(v_1,\lambda_2v_2 \right )=v_1^\dag\lambda_2v_2=\lambda_2 v_1^\dag v_2
\]
\[
\left(A^\dag v_1,v_2 \right )=\left(A v_1,v_2 \right )=\left(\lambda_1v_1,v_2 \right )=(\lambda_1v_1)^\dag v_2=\lambda_1^*v_1^\dag v_2=\lambda_1v_1^\dag v_2
\]
\[
Since \left(v_1,Av_2 \right )=\left(A^\dag v_1,v_2 \right )\Rightarrow \lambda_2 v_1^\dag v_2=\lambda_1v_1^\dag v_2
\]
\[
(\lambda_2-\lambda_1) v_1^\dag v_2=(\lambda_2-\lambda_1) (v_1, v_2)=0
\]
\[
Since \lambda_2\ne\lambda_1 ,  (v_1, v_2)=0
\]
\section*{d)}
\subsection*{1.A matrix with two non degenerate eigenvalues}

Pauli spin matrix \( \begin{pmatrix} 1 & 0\\ 0& -1 \end{pmatrix}\) . Eigenvalues are given by \( (1-\lambda)(-1-\lambda)=0 \) . Eigenvalues are  \( \lambda=\pm1 \).

\subsection*{2.A matrix with complex conjugate eigenvalues}
\(
\begin{pmatrix} 0 & 1\\ -1& 0 \end{pmatrix} \). Eigenvalues are given by \( \lambda^2+1=0 \) . Eigenvalues are\( \lambda=\pm1\).
\subsection*{3.A matrix which cannot be diagonalized}
\(
\begin{pmatrix} 0&0 \\ 0&1 \end{pmatrix}
\)
\subsection*{4.A unitary matrix}

Pauli spin matrices are unitary
\(
\begin{pmatrix} 0 & -i\\ i& 0 \end{pmatrix}, \begin{pmatrix} 0 & 1\\ 1& 0 \end{pmatrix} , \begin{pmatrix} 1 & 0\\ 0& -1 \end{pmatrix}
\)
\end{document}
